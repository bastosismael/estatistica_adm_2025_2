\documentclass{article}
\usepackage[shortlabels]{enumitem}
\usepackage{float}
\usepackage{amsfonts}
\usepackage{amsmath}
\usepackage{amssymb}
\usepackage{amsthm}

\title{Lista de Revisão - Estatística - 2025.1}
\begin{document}
\date{}
\maketitle

1) Sejam definidos os seguintes conjuntos

$$A_1 = \{1,2,3,4,5,6,7,8,9\}, A_2 = \{2, 4, 6, 8\}, A_3 = \{1, 3, 5, 7, 9\}$$
$$A_4 = \{\}, A_5 = \{A, B, C, D\}, A_6 = \{Amarelo, Azul, Verde, Vermelho\}, A_7 = \emptyset, A_8 = \{\emptyset\}$$

Determine

\begin{enumerate}[a)] % a), b), c), ...
    \item $A_1 \cup A_5$
    \item $A_1 \cap A_2$
    \item $A_1 \cup A_4$
    \item $A_1 \cap A_4$
    \item $(A_1 \cup A_6) \cap A_2$
    \item $(A_2 \cup A_3) \cap A_1$
    \item $\bigcup\limits_{i=1}^{3} A_i$
    \item $\bigcap\limits_{i=1}^{3} A_i$
    \item $\bigcap\limits_{i=1}^{7} A_i$
    \item $|A_3| + |A_2| + |A_1|$
    \item $|A_4| + |A_6| + |A_7|$
\end{enumerate}
\vspace{10px}

2) Verifique se as seguintes afirmações são verdadeiras ou falsas. 

\begin{enumerate}
    \item $A_7$ é igual a $A_4$
    \item $A_7$ é igual a $A_8$
    \item $2 \in A_2$
    \item $\{2\} \in A_2$
    \item $A_2 \subset A_1$
    \item $A_3 \subset A_1$
\end{enumerate}

3) Utilizando os conjuntos da questão 1, quando necessário, determine o valor numérico das seguintes expressões:

\begin{enumerate}[a)] % a), b), c), ...
    \item $\sum\limits_{i = 1}^3 i$ \\
    \item $\sum\limits_{i = 1}^5 i^2$ \\
    \item $\sum\limits_{i = 1}^4 \dfrac{1}{i}$ \\
    \item $\sum\limits_{i \in A_1} i$ \\
    \item $\sum\limits_{i \in A_2} i$ \\
    \item $\prod\limits_{i=1}^4 i$
    \item $\prod\limits_{i=1}^4 \dfrac{1}{i^2}$
\end{enumerate}
\vspace{10px}

4) Suponha o conjunto universo $S$ como sendo $S= \{1,2,3,4,5,6,7,8,9,10\}$. Sejam $A=\{2,3,4\}$, $B=\{3,4,5\}$ e $C=\{5,6,7\}$. Explicite os elementos dos seguintes conjuntos:

\begin{enumerate}[a)] % a), b), c), ...
    \item $A^c \cap B$
    \item $A^c \cup B$
    \item $(A^c \cap B^c)^c$
    \item $(A \cap (A^c \cap B^c)^c)^c$
    \item $(A \cap (B \cup C))^c$
\end{enumerate}
\vspace{10px}

5) Suponha que o conjunto universo $S$ seja dado por $S=\{x | 0 \leq x \leq 2\}$. Sejam os conjuntos $A$ e $B$ definidos da seguinte forma: $A = \{x| 1/2 < x \leq 1\}$ e $B=\{x|1/4 \leq x < 3/2\}$. Descreva os seguintes conjuntos:

\begin{enumerate}[a)] % a), b), c), ...
    \item $(A \cup B)^c$
    \item $A \cup B^c$
    \item $(A \cap B)^c$
    \item $A^c \cap B$
\end{enumerate}

\vspace{10px}


6) Verifique as seguintes igualdades envolvendo conjuntos:

\begin{enumerate}[a)] % a), b), c), ...
    \item $(A\cup B)\cap (A \cup C) = A \cup (B \cap C)$
    \item $(A \cup B) = (A \cap B^c)\cup B$
    \item $A^c \cap B = A \cup B$
    \item $(A\cup B)^c \cap C = A^c \cap B^c \cap C^c$
    \item $(A \cap B) \cap (B^c \cap C) = \emptyset$
\end{enumerate}
\vspace{10px}

7) Verifique as seguintes relações:

\begin{enumerate}[a)] % a), b), c), ...
    \item $A \subset B$ e $B \subset C$ implica que $A \subset C$
    \item $A \subset B$ implica que $A\cap B = A$
    \item $A \subset B$ implica que $B^c \subset A^c$
    \item $A \cap B = \emptyset $ e $ C \subset A$ implicam que $B \cap C = \emptyset$ 
\end{enumerate}

8) Escreva o valor numérico das seguintes expressões:

\begin{enumerate}[a)] % a), b), c), ...
    \item $\dfrac{10!}{8!}
    \item $\binom{5}{3}$
    \item $\binom{5}{0}$
    \item $\binom{5}{5}$
\end{enumerate}

9) Escreva em palavras o que significa $\binom{n}{k}$
\end{document}