\documentclass{article}
\usepackage[shortlabels]{enumitem}
\usepackage{float}
\usepackage{minted}

\title{Lista 2 - Estatística - 2025.1}
\begin{document}
\date{}
\maketitle

1) Escreva uma função no R que receba um número natural $n$ e retorne $n!$ (n fatorial). \textbf{Dica:} Utilize os conceitos que vimos sobre estruturas de repetição. 

\vspace{5px}

2) Escreva uma função no R que receba um número natural $n\geq 1$ e retorne o valor que ocupa a $n$-ésima posição da sequência de Fibonacci. 
\textbf{Dica:} Utilize os conceitos que vimos sobre estruturas de repetição. 

\vspace{5px}

3) Realize as seguintes tarefas envolvendo a base de dados adotada no curso:
\begin{enumerate}[a)]
    \item Leia a base de dados do curso dentro do RStudio. 
    \item Crie uma variável chamada $N$ e armazene nela o valor do tamanho do conjunto de dados.
    \item Qual a proporção de indivíduos do Sexo Feminino e do Sexo Masculino nesse conjunto de dados?
    \item Transforme a coluna Frequência para que passe a armazenar o valor da frequência em hertz (Hz) ao invés de batidas por minuto. 
    \item Escreva um código que mostre no console a idade máxima e a mínima existente no conjunto de daddos. 
    \item Crie um histograma para a variável Duracao do Sono e o interprete. 
    \item Crie um gráfico de barras para a variável Ocupacao e o interprete. 
    \item Crie um gráfico de dispersão para as variáveis Nivel de Atividade Fisica (Eixo y) e Idade (Eixo x). Interprete esse gráfico e responda: O que podemos perceber que acontece com a variável Nível de Atividade Física conforme a Idade aumenta? 
\end{enumerate}

\vspace{5px}

4) Considere que a base de dados adotada no curso represente uma população que desejamos estudar.
Dessa população, selecione uma amostra de tamanho $60$ utilizando a Amostragem Aleatória Estratificada com base na variável Sexo usando a abordagem proporcional. 


\textbf{Dica: } Para filtrar o conjunto de dados baseado em uma condição, primeiro devemos escrever o nome do DataFrame acompanhado da condição dentro de colchetes
 sucedido por virgula e pela coluna que se deseja obter ao final da filtragem. Caso queiramos obter todas as colunas, deixamos vazio. 

Exemplo: Suponha que tenhamos interesse em filtrar, no nosso conjunto de dados, todos os indivíduos que tenham idade inferior a 45 anos. O código para isso seria:

\begin{minted}[linenos=false,breaklines]{R}
    dados = read.csv(file.choose()) # Lendo a base de dados do curso.
    dados_l45 = dados[dados$Idade < 45, ] # DataFrame contendo apenas indivíduos com idade inferior a 45 anos.
\end{minted}

Perceba que, após a virgula, não é colocado nada, deixando um espaço vazio. Isso faz com que seja retornada todas as colunas do conjunto de dados filtrado. 
Por outro lado, caso escrevêssemos o código a seguir, obteriamos apenas a Ocupacao dos indivíduos com Idade inferior a 45 anos. 

\begin{minted}[linenos=false,breaklines]{R}
    dados = read.csv(file.choose()) # Lendo a base de dados do curso.
    dados_l45 = dados[dados$Idade < 45, "Ocupacao"] # Coluna Ocupacao contendo apenas indivíduos com idade inferior a 45 anos.
\end{minted}

Uma outra possibilidade para fazer filtros em DataFrames é usar a função \textit{subset}. A função \textit{subset} recebe dois argumentos, o primeiro é o DataFrame no qual
deseja-se aplicar o filtro e o segundo é a condição do filtro. Nesse caso, se quiséssemos filtrar nosso DatafFrame de forma a obter apenas os indivíduos que possuem 
Idade inferior a 45, executaríamos o seguinte código:

\begin{minted}[linenos=false,breaklines]{R}
    dados = read.csv(file.choose()) # Lendo a base de dados do curso.
    dados_l45 = subset(dados, Idade < 45) # DataFrame contendo apenas indivíduos com idade inferior a 45 anos.
\end{minted}

Note que a condição passada usa o nome da coluna sem a necessidade de usar o \$. Isso ocorre pois a própria função subset extrai os nomes das colunas internamente. 
\end{document}