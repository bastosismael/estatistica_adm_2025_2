
\documentclass[12pt]{article}
\usepackage[shortlabels]{enumitem}
\usepackage{float}
\usepackage{xcolor}
\usepackage[a4paper, top=2.5cm, bottom=3cm, left=3cm, right=3cm]{geometry}
\usepackage{parskip}
\usepackage{setspace}
\usepackage{lmodern}
\usepackage[T1]{fontenc}
\usepackage[brazil]{babel}
\usepackage{hyperref}

\title{\textbf{Trabalho 1 — Estatística (2025.1)}}
\author{}
\date{}

\begin{document}
\maketitle
\onehalfspacing

\section*{Objetivo do Trabalho}

Neste trabalho, vocês assumirão o papel de pesquisadores responsáveis por elaborar um texto (semelhante a um artigo científico) com base em dados reais.

O objetivo é aplicar os conceitos estudados até aqui na disciplina, com foco em análise exploratória de dados. Espera-se que vocês elaborem perguntas relevantes sobre o tema e as respondam ao longo do texto, com base na análise do conjunto de dados fornecido.

\section*{Organização e Grupos}

\begin{enumerate}
    \item Cada aluno deve colocar seu nome e e-mail no arquivo de seleção de temas.
    \item Os grupos terão tamanho variável, dependendo do tamanho do conjunto de dados.
    \item Após o prazo de manifestação de interesse, enviarei por e-mail a cada grupo um conjunto de dados referente ao tema escolhido e o método de amostragem. 
    \item Para fins didáticos, trataremos esse conjunto de dados como a população de estudo.
\end{enumerate}

\section*{Avaliação}

O trabalho será avaliado com nota de 0 a 10. As situações abaixo acarretarão automaticamente nota 0:

\begin{itemize}
    \item Entrega fora do prazo.
    \item Trabalho desorganizado ou mal estruturado.
    \item Não utilizar a linguagem R.
    \item Não enviar todos os arquivos solicitados.
    \item Não seguir o método de amostragem informado no e-mail. 
    \item Não preencher o arquivo de seleção de temas com nome e e-mail no prazo indicado.
    \item Plágio. 
\end{itemize}


\section*{Datas e Prazos}
\begin{itemize}
    \item Manifestação do tema de interesse: 22:00h de 25/04 até 30/04 00:00h.
    \item Entrega do trabalho: até 06/06 00:00h.
    \item A entrega deve ser feita no e-mail: ismael@dme.ufrj.br usando a seguinte estrutura:
    \item \begin{itemize}
        \item Assunto: Trabalho de Estatística. 
        \item O corpo do texto deve conter os nomes dos integrantes e o título do trabalho. 
        \item Em anexo devem colocar os arquivos requisitados. 
        \item Apenas \textbf{um} dos membros do grupo deve fazer o envio.  
    \end{itemize}
\end{itemize}

\subsection*{Itens obrigatórios para entrega:}

\begin{itemize}
    \item Arquivo contendo a amostra utilizada nas análises. 
    \item Arquivo qmd e pdf com os resultados e análises da pesquisa (Parte 1 do trabalho).
    \item Arquivo qmd e pdf contendo as respostas (Com os gráficos) da Parte 2 do trabalho (Pode ser entregue no mesmo arquivo do anterior).  
\end{itemize}

\section*{Parte 1 — Análise Exploratória de Dados}
O trabalho deve seguir a seguinte estrutura:

\subsection*{Título}
Relacionado à principal pergunta da pesquisa. Pode ser simples, mas deve representar o objetivo do trabalho.

\subsection*{Capítulo 1 — Introdução}
Apresente o tema, os objetivos da pesquisa, definição da pergunta principal que deseja responder através da pesquisa e 
perguntas adicionais caso verifique necessidade. Além disso, a Introdução pode conter qualquer outro aspecto que julgar relevante (como o tamanho da população, contexto da pesquisa, etc.).

\subsection*{Capítulo 2 — Descrição dos Dados}
Liste as variáveis do conjunto de dados e explique o que cada uma representa. Classifique-as como:

\begin{itemize}
    \item Quantitativa (Discreta ou Contínua) ou
    \item Qualitativa (Nominal ou Ordinal)
\end{itemize}

No caso de variáveis qualitativas, descreva os possíveis valores que a variável assume e seus significados.

\subsection*{Capítulo 3 — Metodologia}
Descreva detalhadamente o método de amostragem adotado, evidenciando o tamanho amostral selecionado. Explicite também as vantagens e desvantagens do método de amostragem adotado. 

Observação: Não é necessário explicar o motivo da seleção do tamanho amostral, haja vista que isso será discutido mais para o fim do curso. 

\subsection*{Capítulo 4 — Análise Exploratória}
Utilize gráficos e tabelas de frequência para analisar \textbf{todas} as variáveis do seu conjunto de dados, excetuando-se apenas variáveis como nome ou id.
Você pode (e deve) utilizar os gráficos discutidos em aula: dispersão, barras, histogramas e boxplots.

Interprete os resultados de forma crítica e os use para responder as perguntas iniciais ou gerar novos insights.

Nesta seção também é interessante você analisar relações entre a variável de interesse e as demais variáveis, fazendo uso do gráfico de dispersão e do coeficiente de correlação. 

\subsection*{Capítulo 5 — Conclusão}
Discuta os principais achados da sua análise. Responda a pergunta principal e comente limitações que afetaram a análise.

\textcolor{red}{\textbf{Importante:}} A estrutura apresentada acima é o mínimo exigido. A exploração de mais elementos de maneira criativa será positivamente avaliada.

\section*{Parte 2 — Tamanho Amostral e Propriedades da Média Amostral}

\subsection*{Exercício 1}

\begin{enumerate}[a)]
    \item Escolha uma variável quantitativa da sua população.
    \item Calcule a média populacional e a armazene na variável \texttt{media\_populacional}.
    \item Crie dois conjuntos de dados (vetores) vazios chamados \texttt{media\_amostral} e \texttt{variancia\_media\_amostral}.
    \item Considerando $n$ como sendo cada uma das seguintes proporções do tamanho da população (\% do tamanho da população):\\
    $2\%, 4\%, 6\%, \dots, 98\%$, realize:
        \begin{itemize}
            \item Extraia uma amostra de tamanho $n$ via Amostragem Aleatória Simples.
            \item Calcule a média amostral da variável selecionada.
            \item Calcule a variância das médias amostrais. 
            \item Armazene a média amostral e a variância das médias amostrais nos vetores \texttt{media\_amostral} e \texttt{variancia\_media\_amostral}, respectivamente.
        \end{itemize}
    \item Construa um gráfico constituído dos seguintes elementos:
        \begin{itemize}
            \item Gráfico de dispersão da \texttt{media\_amostral} vs tamanhos amostrais.
            \item Linha horizontal vermelha em \texttt{y = media\_populacional}.
        \end{itemize}
    \item Construa outro gráfico composto pelos seguintes elementos:
    \begin{itemize}
        \item Gráfico de dispersão da \texttt{variancia\_media\_amostral} vs tamanhos amostrais.
        \item Linha horizontal roxa em \texttt{y = 0}.
        \item Obs: Ajuste seu gráfico para que a linha acima seja visível. 
    \end{itemize}
    \item Escreva um texto interpretando os gráficos.
\end{enumerate}

\subsection*{Exercício 2}

\begin{enumerate}[a)]
    \item Escolha (de preferência a mesma do Exercício 1) uma variável quantitativa do sua população.
    \item Calcule a média populacional e a armazene na variável \texttt{media\_populacional}.
    \item Crie um conjunto de dados (vetor) vazio chamado \texttt{media\_amostras}.
    \item Extraia 1.000 amostras de tamanho $40\%$ do tamanho da população utilizando o método de Amostragem Aleatória Simples e calcule a média de cada uma, armazenando-as no vetor \texttt{media\_amostras}.
    \item Crie um vetor \texttt{media\_das\_medias}, onde cada posição $i$ armazena a média dos $i$ primeiros valores de \texttt{media\_amostras}.
    \item Construa um gráfico contendo:
        \begin{itemize}
            \item Um gráfico de dispersão de \texttt{media\_das\_medias} vs número de amostras ($1, 2, ..., 1000$).
            \item Linha horizontal vermelha em \texttt{y = media\_populacional}.
        \end{itemize}
    \item Escreva um texto interpretando o gráfico.
\end{enumerate}

\end{document}
