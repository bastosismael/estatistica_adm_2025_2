\documentclass{article}
\usepackage[shortlabels]{enumitem}
\usepackage{float}
\usepackage{amsfonts}
\usepackage{amsmath}
\usepackage{amssymb}
\usepackage{amsthm}
\usepackage{dsfont}

\title{Lista 2 - Estatística para Administração - 2024.2}
\begin{document}
\date{}
\maketitle

1) Quantas permutações diferentes existem das letras A,B,C,D,E,F, se
\begin{enumerate}[a)] % a), b), c), ...
    \item Se não há nenhuma restrição então temos $6!$ permutações.
    \item Como fixamos a letra A na primeira posição, temos $5!$ permutações.
    \item Nesse caso também temos $5!$ permutações.
    \item Como a primeira e a última letras são fixas, temos $4!$ permutações.
\end{enumerate}

\vspace{5px}

2) Em uma prova, um estudante deve responder exatamente 7 questões de um total de 10 questões. Quantas escolhas ele tem?

$$\binom{10}{7} =\dfrac{10!}{7!3!} = \dfrac{10 \cdot 9 \cdot 8}{3 \cdot 2 \cdot 1} = 120$$
\vspace{5px}

3) No antigo padrão brasileiro, placas de carro eram códigos com 3 letras seguidas por 4 números. Os ministérios envolvidos com transporte e mobilidade nos quatro países do Mercosul resolveram, entretanto, criar um emplacamento comum com um padrão único no país.
\begin{itemize}
    \item Brasil: AAA 1A11
    \item Uruguai: AAA 1111
    \item Argentina e Venezuela: AA 111 AA
    \item Paraguai: AAAA 111
\end{itemize}

Em que, A pode ser substituído por qualquer letra e 1 por qualquer número. Quantas placas diferentes podem ser criadas em cada país seguindo esse padrão?

\begin{itemize}
    \item Brasil: $26^4 \cdot 10^3$
    \item Uruguai: $26^3 \cdot 10^4$
    \item Argentina e Venezuela: $26^4 \cdot 10^3$ 
    \item Paraguai: $26^4 \cdot 10^3$
\end{itemize}
\vspace{5px}

4) Descreva o espaço amostral dos seguintes experimentos

\begin{enumerate}[a)] % a), b), c), ...
    \item $\Omega = \{Cara, Coroa\}^4$
    \item $\Omega= \{0, 1, 2, 3, \dots\}$
    \item $\Omega= [0, \infty)$ 
    \item $\Omega = \{3,4,5,6,7,8,9,10\}$
    \item $\Omega = \{10, 11, 12, 13, \dots \}$
    \item $\Omega = \{preta\}$
    \item $\Omega = \{verde, branca, vermelha\}$ 
    \item $\Omega = \{verde, branca, vermelha\}^2$ 
\end{enumerate}

\vspace{5px}

5) Informações sobre fundos mútuos, fornecidas pela Morningstar Investment Research, incluem o tipo de fundo mútuo (Capital Nacional Norte-Americano, Capital Internacional ou Renda Fixa) e a classificação da Morningstar para o fundo. A classificação vai de 1 estrela (menor classificação) a 5 estrelas (maior classificação). Suponha que uma amostra com 25 fundos mútuos forneceu as seguintes contagens:

\begin{itemize}
    \item 16 fundos mútuos eram de Capital Nacional Norte-Armericano
    \item 13 fundos mútuos foram classificados com 3 estrelas ou menos
    \item 7 dos fundos de Capital Nacional Norte-Americanos foram classificados com 4 estrelas
    \item 2 dos fundos de Capital Nacional Norte-Americano foram classificados com 5 estrelas
\end{itemize}

Assuma que um desses 25 fundos mútuos será aleatoriamente selecionado, a fim de saber mais sobre o fundo em questão e sua estratégia de investimento 

Definamos os seguintes eventos:

 \begin{itemize}
        \item A: O fundo ser de Capital Nacional Norte-Americano.
        \item B: O fundo ser avaliado com 4 ou 5 estrelas. 
    \end{itemize}
    
\begin{enumerate}[a)]
    \item $\mathds{P}(A) = \dfrac{16}{25}$
    \item $\mathds{P}(B) = \dfrac{12}{25}$
    \item $\mathds{P}(A \cap B) = \dfrac{9}{25}$
    \item 
   Temos que:
    \begin{align*}
       \mathds{P}(A\cup B) &= \mathds{P}(A) + \mathds{P}(B) - \mathds{P}(A \cap B) \\
       &= \dfrac{16}{25} + \dfrac{12}{25} - \dfrac{9}{25} = \dfrac{19}{25}
    \end{align*}
\end{enumerate}

\vspace{5px}

6) Uma companhia de seguros analisou a frequência com que 2.000 segurados(1.000 homens e 1.000 mulheres) usaram o hospital. Os resultados são apresentados na tabela a seguir:

\begin{table}[H]
\centering
\begin{tabular}{ccc}
\hline
                      & Homens & Mulheres \\ \hline
Usaram o hospital     & 100    & 150      \\
Não usaram o hospital & 900    & 850      \\ \hline
\end{tabular}
\end{table}

\begin{enumerate}[a)] % a), b), c), .
    \item $\dfrac{250}{2000} = 0,125$
    \item $\dfrac{100}{1000} = 0,10$
    \item $\dfrac{150}{1000} = 0,15$
\end{enumerate}

\vspace{5px}

7) Sejam $A$ e $B$ dois eventos tais que $\mathds{P}(A) = 1/2, \mathds{P}(B)=1/4$ e $\mathds{P}(A\cap B) = 1/5$. Calcule as probabilidades dos seguintes eventos:

\begin{enumerate}[a)] % a), b), c), .
    \item $\mathds{P}(A^c) = 1 - \mathds{P}(A) = 1/2$
    \item $\mathds{P}(B^c) = 1 - \mathds{P}(B) = 3/4$
    \item $\mathds{P}(A \cup B) = \mathds{P}(A) + \mathds{P}(B) - \mathds{P}(A\cap B) = 1/2 + 1/4 - 1/5 = 0,55$
    \item $\mathds{P}(A^c \cap B) = \mathds{P}(B) - \mathds{P}(A \cap B) = 1/4 - 1/5 = 0,05$
    \item $\mathds{P}(A \cap B^c) = \mathds{P}(A) - \mathds{P}(A \cap B) = 1/2 - 1/5 = 0,3$
    \item $\mathds{P}(A^c \cap B^c) =\mathds{P}(A \cup B)^c  = 1 - \mathds{P}(A \cup B) = 1 - 0.55 = 0,45$
    \item $\mathds{P}(A^c \cup B^c) = \mathds{P}(A \cap B)^c =  1 - \mathds{P}(A \cap B) = 1 - 1/5 = 0,8$
\end{enumerate}

\vspace{5px}

8)Em uma escola, 60\% dos estudantes não usam anel nem colar; 20\% usam anel e 30\%
colar. Se um aluno é escolhido aleatoriamente, qual a probabilidade de que esteja usando

Definamos:

\begin{itemize}
    \item A : Usar anel
    \item B: Usar colar
\end{itemize}
\begin{enumerate}[a)] % a), b), c), .
    \item $\mathds{P}(A \cup B) = \mathds{P}(A) + \mathds{P}(B) - \mathds{P}(A \cap B)$
    Note que não temos $\mathds{P}(A \cap B)$, mas temos $\mathds{P}(A^c \cap B^c)$, logo
    \begin{align*}
        \mathds{P}(A^c \cap B^c) &= \mathds{P}(A \cup B)^c\\
        &= 1 - \mathds{P}(A \cup B)\\
        &= 1 - (\mathds{P}(A) + \mathds{P}(B) - \mathds{P}(A \cap B)\\
        &= 1 - (0,2 + 0,3 - \mathds{P}(A \cap B)\\
        &= 1 - 0,5  + \mathds{P}(A \cap B)\\
        &= 0,5  + \mathds{P}(A \cap B)\\
    \end{align*}

    Logo

    $$\mathds{P}(A \cap B) = 0,6 - 0,5 = 0,1$$

    Portanto
    $\mathds{P}(A \cup B) = 0,2 + 0,3 - 0,1 = 0,4$
    \item $\mathds{P}(A \cap B) = 0,1$
    \item $\mathds{P}(A \cap B^c) = \mathds{P}(A) - \mathds{P}(A \cap B) = 0,2 - 0,1 = 0,1$
\end{enumerate}

\vspace{5px}

9)Um restaurante popular apresenta apenas dois tipos de refeições: salada completa ou um prato
à base de carne. Considere que 20\% dos fregueses do sexo masculino preferem a salada, 30\%
das mulheres escolhem carne, 75\% dos fregueses são homens e os seguintes eventos:

\begin{itemize}
    \item H: freguês é homem
    \item M: freguês é mulher
    \item A: freguês prefere salada
    \item B: freguês prefere carne
\end{itemize}

Calcular:

\begin{enumerate}[a)] % a), b), c), .
    \item$ \mathds{P}(H) = 0,75$
    \item $\mathds{P}(A|H) = 0,2$
    \item $\mathds{P}(B|M) = 0,30$
    \item $\mathds{P}(A \cap H) = \mathds{P}(A | H)\mathds{P}(H) = 0,2 \cdot 0,75 = 0,15$
\end{enumerate}

\vspace{5px}

10) Dois dados de 6 faces são lançados e é anotada a face obtida em cada um dos dados (Por exemplo, se saiu o resultado 2 no primeiro e 4 no segundo, é anotado o par (2,4)). Com base nisso, resolva as seguintes questões:
\begin{enumerate}[a)] % a), b), c), .
    \item $\Omega = \{1,2,3,4,5,6\}^2$
    \item Definamos o evento A: O resultado do primeiro dado é igual ao do segundo. Segue que: $A = \{(1,1), (2,2), (3,3), (4,4), (5,5), (6,6)\}$

    Logo, 

    $$\mathds{P}(A) = \dfrac{6}{36} = \dfrac{1}{6}$$
    \item Definamos o evento B: O resultado do primeiro dado é maior que o do segundo. Segue que:

    $$B = \{(2,1), (3,2), (3,1), (4,3), (4,2), (4,1), (5,4), (5,3), (5,2), (5,1), (6,5), (6,4), (6,3), (6,2), (6,1) \}$$

    Logo, 

    $$\mathds{P}(B) = \dfrac{15}{36} = 0,42$$
    \item  Definamos o evento C: O resultado do primeiro dado é maior ou igual ao do segundo. Segue que:

    {\tiny $$C = \{(1,1), (2,2), (2,1), (3,3), (3,2), (3,1), (4,4), (4,3), (4,2), (4,1), (5,5), (5,4), (5,3), (5,2), (5,1), (6,6), (6,5), (6,4), (6,3), (6,2), (6,1) \}$$}

    Logo, 

    $$\mathds{P}(B) = \dfrac{21}{36} = 0,58$$
    
    \item Definamos o evento D: A soma dos resultados da 3. Segue que:

$$D = \{(1,2), (2,1)\}$$

Logo, 

$$\mathds{P}(D) = \dfrac{2}{36}$$
        
\end{enumerate}


\vspace{5px}

11) Uma urna contém 20 bolas vermelhas, 10 bolas azuis e 5 bolas brancas. Um experimento consiste em retirar uma bola da urna, olhar sua cor, descartar a bola e em seguida retirar outra bola e olhar sua cor. Resolva as seguintes questões:

Para a resolução desse problema, consideremos os seguintes eventos:

\begin{itemize}
    \item $V_i:$ A i-ésima bola é vermelha
    \item $A_i:$ A i-ésima bola é azul
    \item $B_i:$ A i-ésima bola é branca
\end{itemize}

Com $i=1,2$
\begin{enumerate}[a)] % a), b), c), .
    \item $\Omega =\{vermelha, azul, branca\}^2$ 
    \item $\mathds{P}(A_1 \cap A_2) = \mathds{P}(A_1) \mathds{P}(A_2|A_1)= \dfrac{10}{35} \cdot \dfrac{9}{34} = \dfrac{90}{1190} = 0,07$
    \item 
    \begin{align*}
        \mathds{P}((B_1 \cap A_2) \cup (A_1 \cap B_2))  &= \mathds{P}(B_1 \cap A_2) + \mathds{P}(A_1 \cap B_2) \\
        &= \mathds{P}(B_1)\mathds{P}(A_2 | B_1) + \mathds{P}(A_1)\mathds{P}(B_2 | A_1)\\
        &= \dfrac{5}{35} \cdot \dfrac{10}{34} + \dfrac{10}{35} \cdot \dfrac{5}{34}\\
        &= \dfrac{50 + 50}{35 \cdot 34} \\
        &= \dfrac{100}{1190} = 0,08
    \end{align*}

    \item \begin{align*}
        \mathds{P}((V_1 \cap V_2) \cup (B_1 \cap B_2)) &= \mathds{P}(V_1 \cap V_2) + \mathds{P}(B_1 \cap B_2)\\
        &=\mathds{P}(V_1) \mathds{P}(V_2 | V_1) + \mathds{P}(B_1) \mathds{P}(B_2 | B_1)\\
        &=\dfrac{20}{35} \cdot  \dfrac{19}{34} + \dfrac{5}{35} \cdot  \dfrac{4}{34}\\
        &=\dfrac{400}{1190} = 0,34
    \end{align*}
\end{enumerate}

12) Considere a mesma situação da questão anterior, mas no entanto assuma que após retirar a primeira bola, ela seja colocada novamente na urna ao invés de ser descartada. Nesse caso, resolva as seguintes questões:

\begin{enumerate}[a)] % a), b), c), .
    \item $\mathds{P}(A_1, A_2) = \dfrac{10}{35}\cdot \dfrac{10}{35} = \dfrac{100}{1225} = 0,08$
    \item \begin{align*}
        \mathds{P}((B_1 \cap A_2) \cup (A_1 \cap B_2))  &= \mathds{P}(B_1 \cap A_2) + \mathds{P}(A_1 \cap B_2) \\
        &= \mathds{P}(B_1)\mathds{P}(A_2 | B_1) + \mathds{P}(A_1)\mathds{P}(B_2 | A_1)\\
        &= \dfrac{5}{35} \cdot \dfrac{10}{35} + \dfrac{10}{35} \cdot \dfrac{5}{35}\\
        &= \dfrac{50 + 50}{35 \cdot 35} \\
        &= \dfrac{100}{1225} = 0,08
    \end{align*}
    \item \begin{align*}
        \mathds{P}((V_1 \cap V_2) \cup (B_1 \cap B_2)) &= \mathds{P}(V_1 \cap V_2) + \mathds{P}(B_1 \cap B_2)\\
        &=\mathds{P}(V_1) \mathds{P}(V_2 | V_1) + \mathds{P}(B_1) \mathds{P}(B_2 | B_1)\\
        &=\dfrac{20}{35} \cdot  \dfrac{20}{35} + \dfrac{5}{35} \cdot  \dfrac{5}{35}\\
        &=\dfrac{425}{1225} = 0,35
    \end{align*}
\end{enumerate}

13) Uma moeda é viciada de modo que a probabilidade de sair cara é 4 vezes maior que a de sair coroa. Para 2 lançamentos independentes dessa moeda, determine:

Para resolver esse exercício, primeiro precisamos determinar numericamente a probabilidade de sair cara e coroa. 

Definamos os seguintes eventos:

\begin{itemize}
    \item A: Sair cara
    \item B: Sair coroa
\end{itemize}

Sabemos que:

$$\mathds{P}(A) = 4 \cdot \mathds{P}(B)$$

Além disso, 

$$\mathds{P}(A) + \mathds{P}(B) = 1$$

Logo:

$$4 \cdot \mathds{P}(B) + \mathds{P}(B) = 1$$

Portanto, 

$$\mathds{P}(B) = \dfrac{1}{5}$$
$$\mathds{P}(A) = \dfrac{4}{5}$$

\begin{enumerate}[a)] % a), b), c), .
    \item $\Omega = \{(Cara, Coroa\}^2$
    \item $$\mathds{P}((Cara, Coroa) \cup (Coroa, Cara)) = \dfrac{4}{5}\cdot \dfrac{1}{5} + \dfrac{1}{5}\cdot\dfrac{4}{5} = \dfrac{8}{25} = 0,32$$
    \item $$\mathds{P}((Cara, Coroa) \cup (Coroa, Cara) \cup (Cara, Cara)) = \dfrac{4}{5}\cdot \dfrac{1}{5} + \dfrac{1}{5}\cdot\dfrac{4}{5} + \dfrac{4}{5}\cdot \dfrac{4}{5} = \dfrac{24}{25} = 0,960$$
    \item $$\mathds{P}((Cara, Cara), (Coroa, Coroa)) =\dfrac{4}{5} \cdot \dfrac{4}{5} + \dfrac{1}{5} \cdot \dfrac{1}{5} = \dfrac{17}{25} = 0,68$$
\end{enumerate}



14)Peças produzidas por uma máquina são classificadas como defeituosas,
recuperáveis ou perfeitas com probabilidade de 0,1; 0,2 e 0,7; respectivamente.
De um grande lote, foram sorteadas duas peças com reposição. Calcule:

Definamos os seguintes eventos:

\begin{itemize}
    \item D: A peça ser defeituosa
    \item R: A peça ser recuperável
    \item P: A peça ser defeituosa
\end{itemize}

$\Omega=\{D, R, P\}^2$

\begin{enumerate}[a)] % a), b), c), .
    \item $\mathds{P}(D,D) = 0,1^2 = 0,01$
    \item $1 - \mathds{P}(P^c, P^c)= 0,91$
    \item $\mathds{P}(R,P) + \mathds{P}(P,R) = 0,2 \cdot 0,7 + 0,7 \cdot 0,2 = 0,28$
    \item Supomos que as retiradas são independentes. Nesse caso deveríamos ter as probabilidades condicionais. 
\end{enumerate}

15) Suponha que tenhamos dois eventos, A e B, que sejam mutuamente exclusivos. Suponha, além disso, que saibamos que $\mathds{P}(A) = 0,30$ e $\mathds{P}(B) = 0,40$.

\begin{enumerate}[a)] % a), b), c), .
    \item $0$
    \item $0$
    \item Esse estudando está equivocado, se isso fosse verdade teríamos que a resposta da questão b seria $\mathds{P}(A)\mathds{P}(B)$ 
    \item Dois eventos mutuamente exclusivos são necessariamente dependentes um do outro, pois a ocorrência de um inviabiliza a ocorrência do outro. 
\end{enumerate}


\end{document}