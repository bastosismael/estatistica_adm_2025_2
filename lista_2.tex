\documentclass{article}
\usepackage[shortlabels]{enumitem}
\usepackage{float}
\usepackage{amsfonts}
\usepackage{amsmath}
\usepackage{amssymb}
\usepackage{amsthm}
\usepackage{dsfont}

\title{Lista 2 - Estatística para Administração - 2024.2}
\begin{document}
\date{}
\maketitle

\vspace{5px}

1) Em uma prova, um estudante deve responder exatamente 7 questões de um total de 10 questões. Quantas possíveis escolhas ele tem?

\vspace{5px}


2) Descreva o espaço amostral dos seguintes experimentos

\begin{enumerate}[a)] % a), b), c), ...
    \item Uma moeda é jogada quatro vezes e é observada a sequência obtida de caras e coroas
    \item Em uma linha de produção que fabrica peças em série é feita a contagem de peças defeituosas produzidas em um período de 24 horas. 
    \item Uma lâmpada é fabricada, em seguida é ensaiado o tempo de duração da vida da lâmpada através da colocação em um soquete e anotação do tempo decorrido (em horas) até queimar.
    \item Um lote de 10 peças contém 3 peças defeituosas. As peças são retiradas uma a uma (sem reposição da peça retirada) até que a última peça defeituosa seja encontrada. O número total de peças retiradas do lote é contado. 
    \item Peças são fabricadas até que 10 peças perfeitas sejam produzidas. O número total de peças fabricadas é contado. 
    \item De uma urna que só contém bolas pretas, tira-se uma bola e verifica-se sua cor. 
    \item De uma urna que contém 3 bolas verdes, 4 bolas brancas e 7 bolas vermelhas, retira-se uma bola e verifica-se sua cor. 
    \item De uma urna que contém 3 bolas verdes, 4 bolas brancas e 7 bolas vermelhas, retira-se duas bolas e verifica-se a cor de cada uma delas. 
\end{enumerate}

\vspace{5px}



3) Informações sobre fundos mútuos, fornecidas pela Morningstar Investment Research, incluem o tipo de fundo mútuo (Capital Nacional Norte-Americano, Capital Internacional ou Renda Fixa) e a classificação da Morningstar para o fundo. A classificação vai de 1 estrela (menor classificação) a 5 estrelas (maior classificação). Suponha que uma amostra com 25 fundos mútuos forneceu as seguintes contagens:


Assuma que um desses 25 fundos mútuos será aleatoriamente selecionado, a fim de saber mais sobre o fundo em questão e sua estratégia de investimento 

\begin{enumerate}[a)] % a), b), c), .
    \item Qual é a probabilidade de selecionar um fundo de Capital Nacional Norte-Americano?
    \item Qual é a probabilidade de selecionar um fundo com uma classificação de 4 ou 5 estrelas?
    \item Qual é a probabilidade de selecionar um fundo que seja de Capital Nacional Norte-Americano e que tenha sido classificado com 4 ou 5 estrelas?
    \item Qual é a probabilidade de selecionar um fundo que seja de Capital Nacional Norte-Americano ou que tenha sido classificado com 4 ou 5 estrelas?
\end{enumerate}

\vspace{5px}

4) Uma companhia de seguros analisou a frequência com que 2.000 segurados(1.000 homens e 1.000 mulheres) usaram o hospital. Os resultados são apresentados na tabela a seguir:

\begin{table}[H]
\centering
\begin{tabular}{ccc}
\hline
                      & Homens & Mulheres \\ \hline
Usaram o hospital     & 100    & 150      \\
Não usaram o hospital & 900    & 850      \\ \hline
\end{tabular}
\end{table}

\begin{enumerate}[a)] % a), b), c), .
    \item Qual a probabilidade de que uma pessoa segurada use o hospital?
    \item Qual a probabilidade de que uma pessoa segurada use o hospital dado que ela é homem? 
    \item Qual a probabilidade de que uma pessoa segurada use o hospital dado que ela é mulher? 
\end{enumerate}

\vspace{5px}

5) Sejam $A$ e $B$ dois eventos tais que $\mathds{P}(A) = 1/2, \mathds{P}(B)=1/4$ e $\mathds{P}(A\cap B) = 1/5$. Calcule as probabilidades dos seguintes eventos:

\begin{enumerate}[a)] % a), b), c), .
    \item A não ocorre
    \item B não ocorre
    \item Pelo menos um de A e B ocorre. 
    \item A não ocorre e B sim.
    \item B não ocorre e A sim. 
    \item Não ocorre nenhum de A e B.
    \item Pelo menos um de A e B não ocorre. 
\end{enumerate}

\vspace{5px}

6)Em uma escola, 60\% dos estudantes não usam anel nem colar; 20\% usam anel e 30\%
colar. Se um aluno é escolhido aleatoriamente, qual a probabilidade de que esteja usando

\begin{enumerate}[a)] % a), b), c), .
    \item Pelo menos uma das joias
    \item Ambas as joias
    \item Um anel mas não um colar
\end{enumerate}

\vspace{5px}

7)Um restaurante popular apresenta apenas dois tipos de refeições: salada completa ou um prato
à base de carne. Considere que 20\% dos fregueses do sexo masculino preferem a salada, 30\%
das mulheres escolhem carne, 75\% dos fregueses são homens e os seguintes eventos:

\begin{itemize}
    \item H: freguês é homem
    \item M: freguês é mulher
    \item A: freguês prefere salada
    \item B: freguês prefere carne
\end{itemize}

Calcular:

\begin{enumerate}[a)] % a), b), c), .
    \item$ \mathds{P}(H)$
    \item $\mathds{P}(A|H)$
    \item $\mathds{P}(B|M)$
    \item $\mathds{P}(A \cap H)$
\end{enumerate}

\vspace{5px}

8) Dois dados de 6 faces são lançados e é anotada a face obtida em cada um dos dados (Por exemplo, se saiu o resultado 2 no primeiro e 4 no segundo, é anotado o par (2,4)). Com base nisso, resolva as seguintes questões:
\begin{enumerate}[a)] % a), b), c), .
    \item Determine o espaço amostral desse experimento.
    \item Determine a probabilidade do resultado do primeiro dado ser igual ao do segundo.
    \item Determine a probabilidade do resultado do primeiro dado ser maior do que o do segundo. 
    \item Determine a probabilidade do resultado do primeiro ser maior ou igual ao do segundo. 
    \item Determine a probabilidade da soma dos resultados ser igual a 3
\end{enumerate}


\vspace{5px}

9) Uma urna contém 20 bolas vermelhas, 10 bolas azuis e 5 bolas brancas. Um experimento consiste em retirar uma bola da urna, olhar sua cor, descartar a bola e em seguida retirar outra bola e olhar sua cor. Resolva as seguintes questões:

\begin{enumerate}[a)] % a), b), c), .
    \item Determine o espaço amostral desse experimento
    \item Determine a probabilidade de obtermos duas bolas azuis
    \item Determine a probabilidade de obtermos uma bola branca e uma azul
    \item Determine a probabilidade de obtermos duas bolas brancas ou duas bolas vermelhas. 
\end{enumerate}

10) Considere a mesma situação da questão anterior, mas no entanto assuma que após retirar a primeira bola, ela seja colocada novamente na urna ao invés de ser descartada. Nesse caso, resolva as seguintes questões:

\begin{enumerate}[a)] % a), b), c), .
    \item Determine a probabilidade de obtermos duas bolas azuis
    \item Determine a probabilidade de obtermos uma bola branca e uma azul
    \item Determine a probabilidade de obtermos duas bolas brancas ou duas bolas vermelhas. 
\end{enumerate}

13) Uma moeda é viciada de modo que a probabilidade de sair cara é 4 vezes maior que a de sair coroa. Para 2 lançamentos independentes dessa moeda, determine:

\begin{enumerate}[a)] % a), b), c), .
    \item O espaço amostral.
    \item A probabilidade de sair somente uma cara.
    \item A probabilidade de sair pelos menos uma cara.
    \item A probabilidade de dois resultados iguais.
\end{enumerate}


14)Peças produzidas por uma máquina são classificadas como defeituosas,
recuperáveis ou perfeitas com probabilidade de 0,1; 0,2 e 0,7; respectivamente.
De um grande lote, foram sorteadas duas peças com reposição. Calcule:

\begin{enumerate}[a)] % a), b), c), .
    \item Probabilidade de duas peças serem defeituosas
    \item Probabilidade de pelo menos uma peça ser perfeita
    \item Probabilidade de uma peça ser recuperável e uma perfeita
    \item Indique as suposições utilizadas para resolver os itens anteriores. E se o sorteio for sem reposição?
\end{enumerate}

15) Suponha que tenhamos dois eventos, A e B, que sejam mutuamente exclusivos. Suponha, além disso, que saibamos que $\mathds{P}(A) = 0,30$ e $\mathds{P}(B) = 0,40$.

\begin{enumerate}[a)] % a), b), c), .
    \item Quanto é $\mathds{P}(A \cap B)$
    \item Quanto é $\mathds{P}(A | B)$
    \item Um estudante de estatística argumenta que os conceitos de eventos mutuamente exclusivos e eventos independentes são, na verdade, os mesmos, e que se os eventos são mutuamente exclusivos eles devem ser independentes. Você concorda com esta afirmação? Use as informações sobre probabilidade neste problema para justificar sua resposta. 
    \item Qual a conclusão geral você tiraria a respeito dos eventos mutuamente exclusivos e dos eventos independentes em razão dos resultados deste problema?
\end{enumerate}





\end{document}