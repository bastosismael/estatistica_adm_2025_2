\documentclass{article}
\usepackage[shortlabels]{enumitem}
\usepackage{float}
\usepackage{amsfonts}
\usepackage{amsmath}
\usepackage{amssymb}
\usepackage{amsthm}

\title{Lista de Revisão - Estatística para Administração - 2024.2}
\begin{document}
\date{}
\maketitle

1) Sejam definidos os seguintes conjuntos

$$A_1 = \{1,2,3,4,5,6,7,8,9\}, A_2 = \{2, 4, 6, 8\}, A_3 = \{1, 3, 5, 7, 9\}$$
$$A_4 = \{\}, A_5 = \{A, B, C, D\}, A_6 = \{Amarelo, Azul, Verde, Vermelho\}, A_7 = \emptyset, A_8 = \{\emptyset\}$$

Determine

\begin{enumerate}[a)] % a), b), c), ...
    \item  $\{1,2,3,4,5,6,7,8,9,A, B, C, D\}$
    \item $A_2$
    \item $A_1$
    \item $A_4$
    \item $A_2$
    \item $A_1$
    \item $A_1$
    \item $\emptyset$
    \item $\emptyset$
    \item $5 + 4 + 9 = 18$
    \item $0 + 4 + 0 = 4$
\end{enumerate}
\vspace{10px}

2) Verifique se as seguintes afirmações são verdadeiras ou falsas. 

\begin{enumerate}
    \item Verdadeira
    \item Falsa
    \item Verdadeira
    \item Falsa
    \item Verdadeira
    \item Verdadeira
\end{enumerate}

3) Utilizando os conjuntos da questão anterior quando necessário, calcule os seguintes somatórios:

\begin{enumerate}[a)] % a), b), c), ...
    \item $1+2+3 = 6$ \\
    \item $1^2 + 2^2 + 3^2 + 4^2 +5^2 = 55$ \\
    \item $1+ \dfrac{1}{2} + \dfrac{1}{3} + \dfrac{1}{4} = 1 + \dfrac{6 + 4 + 3}{12} = \dfrac{25}{12}$ \\
    \item $1 + 2 + 3 + 4 + 5 +6 + 7 + 8 + 9 = 45$ \\
    \item $2+4+6+8 = 20$ \\
    \item $1 \cdot 2 \cdot 3 \cdot 4 = 24$
    \item $1 \cdot \dfrac{1}{4} \cdot \dfrac{1}{9} \cdot \dfrac{1}{16} = \dfrac{1}{576}$
\end{enumerate}
\vspace{10px}

4) Suponha o conjunto universo $S$ como sendo $S= \{1,2,3,4,5,6,7,8,9,10\}$. Sejam $A=\{2,3,4\}$, $B=\{3,4,5\}$ e $C=\{5,6,7\}$. Explicite os elementos dos seguintes conjuntos:

\begin{enumerate}[a)] % a), b), c), ...
    \item $\{1,5,6,7,8,9,10\} \cap \{3,4,5\} = \{5\}$
    \item$\{1,5,6,7,8,9,10\} \cup \{3,4,5\} = \{1,3,4,5,6,7,8,9,10\}$
    \item $(\{1,5,6,7,8,9,10\} \cap \{1,2,6,7,8,9,10\})^c = \{1,6,7,8,9, 10\}^c = \{2,3,4,5\}$
    \item $(\{2,3,4\} \cap \{2,3,4,5\})^c = \{2,3,4\}^c = \{1,5,6,7,8,9,10\}$
    \item $(\{2,3,4\} \cap \{3,4,5,6,7\})^c = \{3,4\}^c = \{1,2,5,6,7,8,9,10\} $
\end{enumerate}
\vspace{10px}

5) Suponha que o conjunto universo $S$ seja dado por $S=\{x | 0 \leq x \leq 2\}$. Sejam os conjuntos $A$ e $B$ definidos da seguinte forma: $A = \{x| 1/2 < x \leq 1\}$ e $B=\{x|1/4 \leq x < 3/2\}$. Descreva os seguintes conjuntos:

\begin{enumerate}[a)] % a), b), c), ...
    \item $(A \cup B)^c$ = $B^c$ = $\{x | 0 \leq x < 1/4 \text{ ou } 3/4 \leq x \leq 2\}$ 
    \item $A \cup B^c = \{x | 1/2 < x \leq  1 \text{ ou } 0 \leq x < 1/4 \text{ ou } 3/2 \leq x \leq 2\}$
    \item $(A \cap B)^c = \{x| 0 \leq x < 1/2 \text{ ou } 1 < x \leq 2\}$
    \item $A^c \cap B = \{x| 1/4 \leq x \leq 1/2 \text{ ou } 1 < x <3/2\}$
\end{enumerate}

\vspace{10px}


6) Quais relações a seguinte são verdadeiras? Dica: Verifique usando diagramas de Venn

\begin{enumerate}[a)] % a), b), c), ...
    \item Verdadeira
    \item Verdadeira
    \item Falsa
    \item Falsa
    \item Verdadeira
\end{enumerate}
\vspace{10px}

7) Use diagramas de Venn para estabelecer as seguintes relações

\begin{enumerate}[a)] % a), b), c), ...
    \item $A \subset B$ e $B \subset C$ implica que $A \subset C$
    \item $A \subset B$ implica que $A\cap B = A$
    \item $A \subset B$ implica que $B^c \subset A^c$
    \item $A \cap B = \emptyset $ e $ C \subset A$ implicam que $B \cap C = \emptyset$ 
\end{enumerate}

8) 
\begin{enumerate}[a)]
    \item $\dfrac{10!}{8!} = 90$
    \item $\binom{5}{3} = 10$
    \item $\binom{5}{0} = 1$
    \item $\binom{5}{5} = 1$
\end{enumerate}
\end{document}