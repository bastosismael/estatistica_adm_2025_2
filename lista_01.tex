\documentclass{article}
\usepackage[shortlabels]{enumitem}
\usepackage{float}
\usepackage{minted}

\title{Lista 1 - Estatística - 2025.1}
\begin{document}
\date{}
\maketitle

1) Analise os códigos abaixo. Qual o valor da variável $x$ ao final da execução de todas as linhas? Tente pensar inicialmente sem rodar o código no RStudio,
em seguida use o RStudio para verificar sua resposta. 
\begin{enumerate}[a)]
    \item 
    
\begin{minted}[
    frame=lines,  % Adds a top and bottom border
    framesep=5pt, % Space between the code and the border
    rulecolor=\color{black}, % Border color
    bgcolor=lightgray!20  % Background color (optional)
  ]{R}
    x = 100
    x = x/4
    if(x <= 30){
        y = x*2
        x = y/5
    }else{
        x = 0
    }
    x
    \end{minted}

    \item 
    
\begin{minted}[
    frame=lines,  % Adds a top and bottom border
    framesep=5pt, % Space between the code and the border
    rulecolor=\color{black}, % Border color
    bgcolor=lightgray!20  % Background color (optional)
  ]{R}
    x = 100
    x = x/4
    if(x <= 30){
      y = x*2
      x = y/5
    }else{
      x = 0
    }
    g = function(x){
      return(x+1)
    }
    x = g(1)
    x
    \end{minted}
    \vspace{14px}
    \item
    \begin{minted}[
        frame=lines,  % Adds a top and bottom border
        framesep=5pt, % Space between the code and the border
        rulecolor=\color{black}, % Border color
        bgcolor=lightgray!20  % Background color (optional)
      ]{R}

    x = 100
    x = x/4
    if(x <= 30){
        y = x*2
        x = y/5
    }else{
        x = 0
    }
    g = function(x){
        return(x+1)
    }
    h = function(y){
        return(y^2)
    }
    x = h(g(x))
    x
    \end{minted}
\item

\begin{minted}[
frame=lines,  % Adds a top and bottom border
framesep=5pt, % Space between the code and the border
rulecolor=\color{black}, % Border color
bgcolor=lightgray!20  % Background color (optional)
]{R}

    k = 20
    l = 40
    s = function(y, z){
        z = y +4
        y = z + 5
        return(z + y)
    }
    x = s(k, l)
    x
\end{minted}

\end{enumerate}


2) Os códigos a seguir apresentam erros. Identifique os erros e explique o motivo de cada um deles. Tente inicialmente identificá-los sem o auxilio do RStudio. 

\begin{enumerate}[a)]

    \item
\begin{minted}[
frame=lines,  % Adds a top and bottom border
framesep=5pt, % Space between the code and the border
rulecolor=\color{black}, % Border color
bgcolor=lightgray!20  % Background color (optional)
]{R}
    d = 10 * k
    k = 2
    d
\end{minted}
        
        \item
\begin{minted}[
frame=lines,  % Adds a top and bottom border
framesep=5pt, % Space between the code and the border
rulecolor=\color{black}, % Border color
bgcolor=lightgray!20  % Background color (optional)
]{R}
    notas_p1 = c(9, 8.5, 4.2, 3.4 7.2)
    notas_trabalho = c(3.4, 2.3, 7.2, 7, 8.5 )
    media_parcial = notas_p1 * 0.4 + notas_trabalho * 0.1
    media_parcial  
\end{minted}

    \item 
    
\begin{minted}[
frame=lines,  % Adds a top and bottom border
framesep=5pt, % Space between the code and the border
rulecolor=\color{black}, % Border color
bgcolor=lightgray!20  % Background color (optional)
]{R}
    aluno = "Maria"
    idade = 29
    sexo = "F"
    if(sexo == "F"){
        mensagem = "Bem vinda ao curso de Estatística da UFRJ"
    }
    else{
        mensagem = "Bem vindo ao curso de Estatística da UFRJ"
    }
    mensagem
\end{minted}

    \item 
\begin{minted}[
    frame=lines,  % Adds a top and bottom border
    framesep=5pt, % Space between the code and the border
    rulecolor=\color{black}, % Border color
    bgcolor=lightgray!20  % Background color (optional)
    ]{R}
    m = 23
    if(m != 23.0){
        s = 40
    }
    s
\end{minted}
\end{enumerate}


3) Crie uma função em R que receba um número inteiro e retorne se esse número é par ou impar. 
\textbf{Dica:} O operador $\%\%$ no R retorna o resto da divisão inteira. Por exemplo, 
executando $4\%\%2$ no R, recebemos o valor 0, que é o resto da divisão de $4$ por $2$. Se escrevermos  $5\%\%2$, recebemos como resposta o valor $1$. 

\vspace{5px}

4) Crie uma função em R que receba um número inteiro não-negativo e retorne a sua raiz quadrada. 

\textcolor{red}{Importante:} A questão 4 espera um número inteiro positivo como entrada, mas nada impede que um valor fora do domínio da função seja fornecido. 
Nesse caso, adicione uma etapa para verificar se o valor inserido é de fato um inteiro não-negativo, caso não seja, retorne a 
mensagem "Erro: Era esperado um inteiro não-negativo". Assuma que não há a possibilidade de serem fornecidos valores racionais e irracionais para a função.).
\vspace{5px}

5) Crie as seguintes funções em R:

\begin{enumerate}[a)]
    \item Uma função chamada \textit{soma} que receba dois números e retorne a soma deles.
    Exemplo: \textit{soma}$(2,4)$  deve retornar $6$.
    \item Uma função chamada \textit{subtr} que receba dois números e retorne a subtração do primeiro pelo segundo.
    Exemplo: \textit{subtr}$(2,4)$  deve retornar $-2$.
    \item Uma função chamada \textit{mult} que receba dois números e retorne a multiplicação entre eles.
    Exemplo: \text{mult}$(2,4)$  deve retornar $8$.
    \item Uma função chamada \textit{div} que receba dois números e retorne a divisão do primeiro pelo segundo. (\textcolor{red}{Atenção:} 
    Perceba que no caso da divisão, temos problema caso o denominador seja igual a 0. Entenda como o R trata esse caso e escreva sua função de tal
     forma que caso o denominador seja 0, então seu código retorne um texto com a mensagem "Erro: Divisão por 0")

    Exemplo: \text{div}$(2,4)$  deve retornar $0.5$.
    \item Crie uma função chamada \textit{calculadora} que recebe como entrada dois números e a operação desejada e retorne o resultado. 
    \textbf{Dica:} Use as funções criadas nos itens anteriores. 
    
    Exemplos: 
    \begin{itemize}
        \item \textit{calculadora}$(2,4, "+")$ deve retornar $6$.
        \item \textit{calculadora}$(2,4, "-")$ deve retornar $-2$.
        \item \textit{calculadora}$(2,4, "*")$ deve retornar $8$.
        \item \textit{calculadora}$(2,4, "/")$ deve retornar $0.5$.
    \end{itemize}
   
\end{enumerate}


\end{document}