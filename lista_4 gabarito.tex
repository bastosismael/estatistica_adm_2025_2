\documentclass{article}
\usepackage[shortlabels]{enumitem}
\usepackage{float}
\usepackage{amsmath}
\usepackage{amsfonts}
\usepackage{dsfont}
\title{Lista 4 - Estatística - 2025.1}
\begin{document}
\date{}
\maketitle

1) 
\begin{enumerate}[a)]
    \item Bernoulli. Citar o que caracteriza uma variável que segue a distribuição de Bernoulli.
    \item Binomial. $Y\sim Binomial(10, 0,1)$
    \item $0,0574$
    \item $1 - 0,9^{10}$
    \item $10$
\end{enumerate}

\vspace{5px}

2) 
\begin{enumerate}[a)]
    \item $\mathbb{E}[Y] = \sum_{i=1}^{20} \mathbb{E}[X_i] = 20 \cdot 0,5 = 10$ e $Var(Y) = \sum_{i=1}^{20} Var(X_i) = 20 \cdot 0,25 = 5$
    \item $W \sim Bernoulli(0,5^2)$; $p_W(w) = 0,5^{2w} (1-0,5^2)^(1-w)$ 
    \item $V \sim Bernoulli(0,5^n)$; $p_V(v) = 0,5^{n \cdot v} (1-0,5^n)^(1-v)$
    \item $supp(X) = \{-2, 2\}$. Média: $0$, Variância: $4$
\vspace{5px}

3) 

\begin{enumerate}[a)]
    \item $S \sim Poisson(3)$
    \item $\dfrac{e^{-3}3^2}{2!}$
    \item $e^{-3/4}$
    \item $1 - \mathds{P}(S\leq 4) = $
    \item $\dfrac{e^{-6}6^6}{6!}$
\end{enumerate}

\vspace{5px}

4) 

\begin{enumerate}[a)]
    \item $3/32$
    \item $11/16$
    \item $3; 0,795$
    \item $3$
\end{enumerate}

\vspace{5px}

5) Seja $Z$ uma variável aleatória tal que $Z\sim N(0,1)$, calcule:

\begin{enumerate}[a)]
    \item $\mathds{P}(Z>1) = 0,1587$
    \item $\mathds{P}(-1 \leq Z<1) = 0,6826$
    \item  $\mathds{P}(Z>2| Z> 1) = 0,1437$
    \item $\mathds{P}(Z>2| Z< -1) = 0$
    \item $\mathds{P}(Z>2| Z< 3) = 0,0215$
    \item $\mathds{P}(Z<-2) = 0,0228$
    \item $\mathds{P}(-4<Z<4) \approx 1$
\end{enumerate}

\vspace{5px}

6) Se X é uma variável aleatória normal com parâmetros $\mu=10$ e $\sigma^2=36$, calcule:

\begin{enumerate}[a)]
    \item $\mathds{P}(X>5) = 0,7967$
    \item $\mathds{P}(4<X<16) = 0,6826$
    \item $\mathds{P}(X<8) = 0,3707$
    \item $\mathds{P}(X<20) = 0,9525$
    \item $\mathds{P}(X>16) = 0,1587$
    \item $\mathds{P}(X>5 | X> 2) = 0,8788$
    \item $x$ tal que $\mathds{P}(X>x) = 0.75 \implies x \approx 5,98$
    \item $\mathds{P}(|X-10| \leq 5) = 0,5934$
\end{enumerate}

\vspace{5px}

7)  Suponha que a altura dos homens de 25 anos de idade, em cm, seja uma variável aleatória normal com parâmetros $\mu = 180$ e $\sigma^2=16$. 
Qual a probabilidade de um homem de 25 anos de idade ter mais de 1,88 de altura?

Definamos $X:$ Altura dos homens de 25 anos em cm. 
$$X \sim Normal(180, 16)$$

$$\mathds{P}(X > 188) = \mathds{P}(Z > 2) = 0,0228$$

\vspace{5px}

8) 
\begin{enumerate}[a)]
    \item $0,0228$
    \item $0,6826$
\end{enumerate}
\vspace{5px}

9) 

\begin{enumerate}[a)]
    \item $X \sim Normal(50, 1,5^2)$
    \item $0,9544$
    \item  $1-0,9544$
    \item $(1-0,9544) \cdot 10000$
    \item $0,9974$
\end{enumerate}

\vspace{5px}


10)
\begin{enumerate}[a)]
    \item IC(\mu, 90\%) = [17,016; 18,984] . Interpretação: É possível afirmar com 90\% de confiança que o intervalo contém a verdadeira média. (Aprofundar no que significa essa frase, conforme visto na  última aula)
    \item IC(\mu, 95\%) = [16,824; 19,176] . Interpretação: É possível afirmar com 95\% de confiança que o intervalo contém a verdadeira média.  (Aprofundar no que significa essa frase, conforme visto na  última aula)
    \item  IC(\mu, 90\%) = [17,412; 18,588] . Interpretação: É possível afirmar com 90\% de confiança que o intervalo contém a verdadeira média.  (Aprofundar no que significa essa frase, conforme visto na  última aula)
\end{enumerate}

11) 

\begin{enumerate}[a)]


    \item $286/625 = 0,4576$

    \item   Usando o método conservador.  $IC(p, 95\%) = [0,4184;  0,4968]

    \item Não.

    \item $2401$
\end{enumerate}

\end{document} 