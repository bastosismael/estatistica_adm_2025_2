\documentclass{article}
\usepackage[shortlabels]{enumitem}
\usepackage{float}
\usepackage{amsmath}
\usepackage{amsfonts}
\usepackage{dsfont}
\title{Lista 5 - Estatística para Administração - 2024.2}
\begin{document}
\date{}
\maketitle

1) Uma empresa produtora café acredita que o peso de suas embalagens de café se comporta como uma variável aleatória Normal de média 200 gramas e desvio padrão 10 gramas. Uma amostra de 25 pacotes é sorteada e pergunta-se:

\begin{enumerate}[a)]
    \item Qual é a probabilidade de que o peso do pacote  não exceda 220 gramas.
    \item Qual o valor de peso a partir do qual a probabilidade do peso ser maior que esse valor seja igual 1\%.

\end{enumerate}

2) Suponha que um hospital deseja saber o peso médio dos recém-nascidos que nasceram no hospital. Para isso, o hospital obtêm uma amostra de 100 bebês utilizando o método de amostragem aleatória simples, obtendo média igual a $4,2$ Kg. O hospital sabe previamente que a variância populacional é igual a 0,16. Nesse cenário, responda as seguintes questões:

\begin{enumerate}[a)]
    \item Qual as vantagens de se usar a amostragem aleatória simples?
    \item Obtenha um intervalo de confiança para a média populacional com 95\% de confiança e interprete-o.
    \item Agora obtenha um intervalo de confiança para a média populacional com 99\% de confiança e interprete-o. O que é possível dizer em relação ao tamanho do intervalo a medida que aumentamos o nível de confiança?
    \item Suponha que o hospital agora tenha conseguido uma amostra de 500 pacientes e obtido a mesma média amostral. Determine o intervalo de confiança para a média populacional com um nível de confiança de 95\%, interprete esse intervalo e diga: Qual a influência do tamanho da amostra sobre o intervalo?
\end{enumerate}

3) Em uma rodovia, os motoristas relataram que, em média, ocorrem 2 acidentes por dia em um determinado trecho. Suponha que o número de acidentes por dia siga uma distribuição de Poisson.


   
    
    
\end{document}