\documentclass{article}
\usepackage[shortlabels]{enumitem}
\usepackage{float}
\usepackage{amsmath}
\usepackage{dsfont}
\title{Lista 3 - Estatística - 2025.1}
\begin{document}
\date{}
\maketitle

1) Um aluno da disciplina de Estatística decide usar a seguinte métrica como medida de resumo para sua análise exploratória de dados:

$$\tilde{x} = \dfrac{\sum_{i=1}^n (x_i - \bar{x})}{n}$$

Mostre que, independentemente de $n$ e do conjunto de dados $x_1, x_2, \dots, x_n$, a métrica escolhida pelo aluno sempre resulta em zero. Faz sentido o aluno adotar essa métrica em sua análise?

\vspace{5px}

2) Mostre que para qualquer conjunto de dados $(x_1, x_2, \dots, x_n)$ com $n \geq 1$, os escores padronizados $z_1, z_2, \dots, z_n$  apresentam média $0$ e desvio padrão igual a $1$. 

\vspace{5px}

3) Discorra sobre as interpretações da probabilidade discutidas em sala de aula, evidenciando as principais críticas que cada uma delas recebe.

\vspace{5px}

4) Descreva um espaço amostral para seguintes experimentos. Diga também sua cardinalidade.

\begin{enumerate}[a)]
    \item Uma moeda é jogada quatro vezes, e é observada a sequência obtida de caras e coroas.
    \item Em uma linha de produção que fabrica peças em série, é feita a contagem de peças defeituosas produzidas em um período de 24 horas.
    \item Uma lâmpada é fabricada e, em seguida, é ensaiado o tempo de duração da vida útil da lâmpada, através da colocação em um soquete e da anotação do tempo (em horas) até que ela queime.
    \item Um lote de 10 peças contém 3 peças defeituosas. As peças são retiradas uma a uma, sem reposição, até que a última peça defeituosa seja encontrada. O número total de peças retiradas do lote é contado.
    \item Peças são fabricadas até que 10 peças perfeitas sejam produzidas. O número total de peças fabricadas é contado.
    \item De uma urna que contém apenas bolas pretas, tira-se uma bola e verifica-se sua cor.
    \item Um dado de 20 faces é jogado 10 vezes.
    \item Um dado de $n$ faces é jogado $k$ vezes.
    \item De uma urna que contém 3 bolas verdes, 4 bolas brancas e 7 bolas vermelhas, retira-se uma bola e verifica-se sua cor.
    \item Da mesma urna anterior, retira-se duas bolas e verifica-se a cor de cada uma delas.
    \item Medir o pH de uma amostra de solução preparada com ácido acético em diferentes diluições. (Assuma que a concentração real do ácido pode variar devido a imprecisões na preparação.)
    \item Verificar se uma reação catalítica ocorre dentro do tempo esperado em um lote de reatores industriais. (Nesse caso, a aleatoriedade está relacionada a variações na pureza dos reagentes, na temperatura ou na eficiência do catalisador.)
    \item Considere o mesmo experimento anterior, mas agora estamos interessados em saber o tempo até que ocorra a reação.
    \item Medir o spin de um elétron em uma direção específica (eixo $z$, por exemplo).
\end{enumerate}

\vspace{5px}

5) Um professor da disciplina de Estatística resolveu adotar a seguinte dinâmica para aplicação de sua prova:

\begin{itemize}
    \item Cada estudante deve responder exatamente 7 questões de um total de 10.
    \item O nível de dificuldade das questões aumenta de acordo com o número da questão, ou seja, a questão 1 é a mais fácil e a 10, a mais difícil.
    \item As questões são de múltipla escolha, contendo, cada uma, 4 possíveis respostas.
    \item As 7 questões são selecionadas de forma aleatória pelo professor no momento em que o aluno entra na sala.
\end{itemize}

Sabendo disso, responda:

\begin{enumerate}[a)]
    \item Quantas possíveis escolhas de questões o professor possui?
    \item Qual a probabilidade de ele escolher as questões mais difíceis, ou seja, exatamente as questões 4, 5, 6, 7, 8, 9 e 10?
    \item Qual a probabilidade de o professor escolher a questão 10?
    \item Qual a probabilidade de o aluno errar todas as questões?
    \item Qual a probabilidade de o aluno acertar pelo menos uma questão?
    \item Qual pressuposto você teve que assumir para responder às questões (d) e (e)? Esse pressuposto é razoável nesse contexto?
\end{enumerate}

6) Um delegado da Polícia Cívil investiga o caso de desaparecimento de uma criança na cidade do Paraná. Sabe-se que no momento do desaparecimento da criança, existiam
mais 5 crianças junto à criança desaparecida, sendo o nome dessas crianças: Judas, Marcos, Matheus, Madalena e Pedro. Com o objetivo de eliminar um possível viés sobre o caso, o 
delegado decide selecionar aleatoriamente duas das 5 crianças para prestar depoimento. Sabe-se que o delegado suspeita que Judas e Pedro podem ter sido
contaminados, ou seja, podem ter sido previamente instruídos a prestarem depoimentos falsos. 
Dessa forma, responda: 

\begin{enumerate}[a)]
    \item Qual a probabilidade de que o delegado acabe selecionando Judas e Pedro para depor?
    \item Qual a probabilidade de que o delegado selecione um grupo que contenha Judas ou Pedro para depor?
\end{enumerate}

\vspace{5px}

7) Considere o experimento do lançamento de dois dados simétricos de 6 faces. Qual
a probabilidade da soma dos resultados do lançamento dos dados dar um número ímpar dado que um dos dados apresentou o valor $1$?

\vspace{5px}

8) Cinquenta e dois por cento dos estudantes de certa faculdade são mulheres. 
Cinco por cento dos estudantes dessa faculdade estão se formando em Engenharia Química.
 Dois por cento dos estudantes são mulheres que estão se formando em Engenharia Química.
  Se um estudante é selecionado aleatoriamente, determine a probabilidade condicional de que:

\begin{enumerate}[a)]
    \item Ele seja mulher, dado que está se formando em Engenharia Química.
    \item Ele esteja se formando em Engenharia Química, dado que é mulher.
\end{enumerate}

\vspace{5px}

9) Uma companhia de seguros analisou a frequência com que 2.000 segurados (1.000 homens e 1.000 mulheres) usaram o hospital. Os resultados são apresentados na tabela a seguir:

\begin{table}[H]
\centering
\begin{tabular}{ccc}
\hline
                      & Homens & Mulheres \\ \hline
Usaram o hospital     & 100    & 150      \\
Não usaram o hospital & 900    & 850      \\ \hline
\end{tabular}
\end{table}

\begin{enumerate}[a)]
    \item Qual a probabilidade de que uma pessoa segurada use o hospital?
    \item Qual a probabilidade de que uma pessoa segurada use o hospital, dado que ela é homem?
    \item Qual a probabilidade de que uma pessoa segurada use o hospital, dado que ela é mulher?
\end{enumerate}


10) Suponha que uma companhia de seguros classifique as pessoas em uma de três classes: risco baixo, risco médio e risco elevado.
Os registros da companhia indicam que as probabilidades de que pessoas com riscos baixo, médio e elevado estejam envolvidas 
em acidentes ao longo de um ano são, respectivamente, 0,05, 0,15 e 0,30. Além disso, 20\% da população é classificada como de risco baixo,
50\% como risco médio e 30\% como risco elevado.

\begin{enumerate}[a)]
    \item Que proporção de pessoas sofre acidentes ao longo de um ano?
    \item Se o segurado A não sofreu acidentes em 2024, qual é a probabilidade de que ele (ou ela) seja uma pessoa de risco baixo ou médio?
\end{enumerate}

\vspace{5px}

11) Sejam $A$ e $B$ dois eventos tais que $\mathds{P}(A) = \dfrac{1}{2}$, $\mathds{P}(B) = \dfrac{1}{4}$ e $\mathds{P}(A\cap B) = \dfrac{1}{5}$. Calcule as probabilidades dos seguintes eventos:

\begin{enumerate}[a)]
    \item A não ocorre.
    \item B não ocorre.
    \item Pelo menos um dos eventos A ou B ocorre.
    \item A não ocorre e B ocorre.
    \item B não ocorre e A ocorre.
    \item Não ocorre nenhum dos eventos A e B.
    \item Pelo menos um dos eventos A e B não ocorre.
\end{enumerate}

\vspace{5px}

12) Seja $A_1, A_2, A_3, \dots, A_{50}$ uma sequência de eventos mutuamente excludentes, tais que:

$$\mathds{P}(A_i) = \dfrac{1}{50}, i \in \{1, 2, 3, \dots, 50 \}$$

Calcule:

\begin{enumerate}[a)]
    \item $\mathds{P}\left(\displaystyle\bigcup_{i=1}^{50} A_i\right)$
    \item $\mathds{P}\left(\left( A_1^c \cap A_2^c \right)^c\right)$
    \item $\mathds{P}\left(\left(\displaystyle\bigcap_{i=1}^{17} A_i^c\right)^c\right)$
    \item O valor de $x$ tal que  $\mathds{P}\left(\left(\displaystyle\bigcap_{i=1}^x A_i^c\right)^c\right) \leq 0,5$
    \item Assuma que $A_1, A_2, A_3, \dots, A_{50}$ são independentes. Calcule $\mathds{P}\left( \displaystyle\bigcap_{i=1}^{50} A_i\right)$
    \item Assuma que $A_1, A_2, A_3, \dots, A_{50}$ são independentes. Encontre $x$ tal que  $\mathds{P}\left( \displaystyle\bigcap_{i=1}^{x} A_i\right) \leq 1,6 \cdot 10^{-7}$
\end{enumerate}

\vspace{5px}

13) Seja $\Omega$ o espaço amostral de um experimento e $A_1, A_2, A_3, \dots, A_{200}$ uma sequência de eventos tais que:

$$A_i = \begin{cases}
\Omega & \text{se $i$ é par} \\
\emptyset & \text{se $i$ é ímpar} 
\end{cases}$$

Calcule:

\begin{enumerate}[a)]
   \item $\mathds{P}\left(\displaystyle\bigcup_{i=1}^{200} A_i\right)$
   \item $\mathds{P}\left(\displaystyle\bigcap_{i=1}^{200} A_i\right)$
\end{enumerate}

\vspace{5px}

14) Em uma turma de 30 alunos, um professor deseja selecionar um grupo de 2 alunos para serem os representantes da turma.
Sabe-se que essa turma possui maior quantidade de pessoas do sexo masculino do que do feminino.
Sabendo disso, responda: Qual é a probabilidade de que o grupo seja formado unicamente por pessoas do sexo feminino?

\vspace{5px}

15) Uma moeda é viciada de modo que a probabilidade de sair cara é 4 vezes maior que a de sair coroa.
 Para dois lançamentos independentes dessa moeda, determine:

\begin{enumerate}[a)]
    \item O espaço amostral.
    \item A probabilidade de sair exatamente uma cara.
    \item A probabilidade de sair pelo menos uma cara.
    \item A probabilidade de saírem dois resultados iguais.
    \item A probabilidade de que o segundo lançamento seja cara, dado que o primeiro foi cara.
\end{enumerate}

\vspace{5px}

16) Suponha que dois eventos, A e B, sejam mutuamente exclusivos. Além disso, $\mathds{P}(A) = 0,30$ e $\mathds{P}(B) = 0,40$.

\begin{enumerate}[a)]
    \item Quanto é $\mathds{P}(A \cap B)$?
    \item Quanto é $\mathds{P}(A | B)$?
    \item Um estudante de Estatística argumenta que os conceitos de eventos mutuamente exclusivos e eventos independentes são, na verdade, os mesmos, e que se os eventos são mutuamente exclusivos, então devem ser independentes. Você concorda com essa afirmação? Use as informações deste problema para justificar sua resposta.
    \item Qual conclusão geral você tira a respeito dos eventos mutuamente exclusivos e dos eventos independentes, com base nos resultados deste problema?
\end{enumerate}

\vspace{5px}

17) Um rei vem de uma família com duas crianças. Qual é a probabilidade de que a outra criança seja uma menina?

\vspace{5px}

18) Três dados simétricos de seis lados são lançados.
\begin{enumerate}[a)]
    \item Defina o espaço amostral desse experimento. Qual a cardinalidade desse espaço amostral?
    \item Sabe-se que o primeiro dado apresentou o valor $6$. Qual a probabilidade de que a soma dos três dados seja igual a $7$?
    \item Qual a probabilidade de que a soma dos dados seja $10$? E de que seja $11$?
    \item Qual a probabilidade de que a soma dos três dados seja igual a $18$, dado que a soma dos dois primeiros foi $12$?
    \item Qual a probabilidade de que a soma dos três dados seja igual a $18$, dado que a soma dos dois primeiros foi $7$?
\end{enumerate}

\vspace{5px}

19) Verifique se as seguintes afirmações são verdadeiras ou falsas:

\begin{enumerate}[a)]
    \item Sejam dois eventos $A$ e $B$. Se $\mathds{P}(A) + \mathds{P}(B) = 1$, então $A$ e $B$ são mutuamente excludentes.
    \item Seja um espaço amostral $\Omega$ e $A$ e $B$ subconjuntos de $\Omega$. Se $A$ e $B$ são independentes, a não ocorrência de $A$ implica na ocorrência de $B$?
    \item Sejam $A$ e $B$ dois eventos. Se $A$ não depende de $B$, então $B$ não depende de $A$?
    \item Sejam $A$, $B$ e $C$ três eventos tais que $\mathds{P}(A|B) = \mathds{P}(A)$ e $\mathds{P}(A|C) = \mathds{P}(A)$. É possível afirmar que $\mathds{P}(A|B \cap C) =  \mathds{P}(A)$?
    \item $\mathds{P}(A \cap B) = 0$ implica que $A$ e $B$ são independentes.
\end{enumerate}

\vspace{5px}

20) Peças produzidas por uma máquina são classificadas como defeituosas,
recuperáveis ou perfeitas com probabilidade de 0,1; 0,2 e 0,7; respectivamente.
De um grande lote, foram sorteadas duas peças com reposição. Calcule:

\begin{enumerate}[a)] % a), b), c), .
    \item Probabilidade de duas peças serem defeituosas
    \item Probabilidade de pelo menos uma peça ser perfeita
    \item Probabilidade de uma peça ser recuperável e uma perfeita
    \item Indique as suposições utilizadas para resolver os itens anteriores. E se o sorteio for sem reposição?
\end{enumerate}



\vspace{5px}

21) Duas fábricas locais, \(A\) e \(B\), produzem rádios. Cada rádio produzido na fábrica \(A\) é defeituoso com probabilidade \(0{,}05\),
enquanto cada rádio produzido na fábrica \(B\) é defeituoso com probabilidade \(0{,}01\).
Suponha que você compre dois rádios que foram produzidos na mesma fábrica, sendo que essa 
é igualmente provável de ser \(A\) ou \(B\). Se o primeiro rádio que você verifica é defeituoso,
qual é a probabilidade condicional de que o outro rádio também seja defeituoso?


\vspace{5px}

22) Uma indústria química realiza o controle de qualidade de lotes de um determinado reagente.
 Para que um lote seja considerado aprovado, ele deve ser aprovado por três testes consecutivos. Caso ele seja reprovado em algum deles, o lote é 
 automaticamente rejeitado.

\begin{itemize}
    \item \textbf{Teste A}: verifica se a pureza está acima de 98\%. A probabilidade de um lote ser aprovado neste teste é de $0{,}95$.
    
    \item \textbf{Teste B}: avalia a estabilidade térmica do composto. Este teste depende da pureza: se o lote foi aprovado no Teste A, a probabilidade de aprovação no Teste B é de $0{,}92$.
    
    \item \textbf{Teste C}: verifica a ausência de contaminantes metálicos. Se o reagente foi aprovado no teste A e B, então a probabilidade de ser aprovado no teste $C$ é $0,95$
\end{itemize}
\begin{enumerate}[a)]
    \item Qual é a probabilidade de que um lote seja aprovado em todos os três testes, ou seja, que ele atenda simultaneamente aos critérios de pureza, estabilidade e ausência de contaminantes?
    \item Se os testes fossem independentes. Qual seria a probabilidade do lote ser aprovado?
\end{enumerate}


\vspace{5px}

23) Uma indústria de materiais produz um tipo específico de polímero utilizando dois tipos de reatores:

\begin{itemize}
    \item Reator Contínuo (R1): responsável por 60\% da produção total.
    \item Reator Batelada (R2): responsável por 40\% da produção total.
\end{itemize}

Sabe-se que a probabilidade de ocorrência de bolhas (um defeito no polímero que compromete sua resistência) varia conforme o tipo de reator:

\begin{itemize}
    \item Se o polímero foi produzido no Reator Contínuo, a probabilidade de apresentar bolhas é de 1,5\%.
    \item Se foi produzido no Reator Batelada a probabilidade de apresentar bolhas é de 4\%.
\end{itemize}

Um lote é selecionado aleatoriamente e verifica-se que ele apresenta bolhas.


Qual é a probabilidade de que esse lote tenha sido produzido no Reator Batelada, dado que apresenta bolhas?

\vspace{5px}

24) Uma planta industrial possui três sensores diferentes (um de temperatura, outro de pressão e outro de vazamento) que monitoram falhas em um processo de destilação.
 Cada sensor é calibrado para detectar um tipo específico de anomalia, mas ocasionalmente gera alarmes falsos.

Sejam definidos os seguintes eventos

\begin{itemize}
    \item A: O sensor de temperatura gera alarme falso. 
    \item B: O sensor de pressão gera alarme falso.
    \item C: O sensor de vazamento gera alarme falso.
\end{itemize}

Sabe-se que: $\mathds{P}(A) = 0{,}03, \mathds{P}(B) = 0{,}02$ e $\mathds{P}(C) = 0{,}05$. Por falta de dados sobre a dependência entre os sensores, a equipe de engenharia deseja estimar uma \textbf{cota superior} para a probabilidade de ocorrer pelo menos um alarme falso durante essa hora.

Determine um limitante superior para a probabilidade de que ocorra pelo menos um alarme falso.

\vspace{5px}

25) Mostre, utilizando apenas a definição de probabilidade condicional que, dados dois eventos $A$ e $B$ tal que $\mathds{P}(A)>0$:

$$\mathds{P}(A|B) = \dfrac{\mathds{P}(B|A)\mathds{P}(A)}{\mathds{P}(B)}$$


\vspace{5px}

26)Mostre que, se \(P(A) > 0\), então:

\[
P(A \cap B \mid A) \geq P(A \cap B \mid A \cup B)
\]

\vspace{5px}

27) Sejam $E, F$ e $G$ três eventos definidos em um mesmo espaço amostral. Verifique, utilizando o Diagrama de Venn, que a seguinte relação é verdadeira:

\begin{equation*}
P(E \cup F \cup G) = P(E) + P(F) + P(G)
- P(E^c \cap F \cap G)`'
- P(E \cap F^c \cap G)
- P(E \cap F \cap G^c)
- 2P(E \cap F \cap G)
\end{equation*}

\vspace{5px}

28) Mostre que, se \(E_1, E_2, \dots, E_n\) são eventos independentes, então:

\[
P(E_1 \cup E_2 \cup \dots \cup E_n) = 1 - \prod_{i=1}^{n} [1 - P(E_i)]
\]

\vspace{5px}

29) Mostre que, se \(P(A \mid B) = 1\), então \(P(B^c \mid A^c) = 1\). Explique com palavras o que esse resultado significa. 

\vspace{5px}

30) Mostre que

\[
\frac{P(A \mid E)}{P(C \mid B)} = \frac{P(A)}{P(C)} \cdot \frac{P( E \mid A)}{P(B \mid C)}
\]


\end{document}