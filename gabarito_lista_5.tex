\documentclass{article}
\usepackage[shortlabels]{enumitem}
\usepackage{float}
\usepackage{amsmath}
\usepackage{amsfonts}
\usepackage{dsfont}
\title{Lista 5 - Estatística para Administração - 2024.2}
\begin{document}
\date{}
\maketitle

1) 

\begin{enumerate}[a)]
    \item Queremos calcular:
    \[
    P(X \leq 220).
    \]
    Primeiro, padronizamos a variável:
    \[
     Z = \frac{220 - 200}{10} = 2.
    \]
    Utilizando a tabela da distribuição normal padrão temos que:
    \[
    P(Z \leq 2) \approx 0.9772.
    \]


    \item 
    \[
    P(X > x) = 0.01 \implies P(X \leq x) = 1 - 0.01 = 0.99.
    \]


    Utilizando a tabela da distribuição Normal fornecida no curso, temos que $P(Z \leq z) = 0.99$ corresponde a $z \approx 2.33$. Assim:
    \[
    \frac{x - 200}{10} = 2.33 \quad \Rightarrow \quad x = 200 + 10 \cdot 2.33 = 223.3.
    \]

\end{enumerate}


2) 

\begin{enumerate}[a)]
    \item Resposta nos slides da disciplina. Importante ressaltar a garantia teórica.  
    \item Sabemos que:
    \begin{itemize}
        \item Média amostral: $\bar{x} = 4,2$ Kg.
        \item Variância populacional: $\sigma^2 = 0,16 \quad \Rightarrow \quad \sigma = \sqrt{0,16} = 0,4$ Kg.
        \item Tamanho da amostra: $n = 100$.
    \end{itemize}
    O intervalo de confiança para a média populacional com um nível de confiança de 95\% é dado por:
    
    $$IC(\mu, 95\%) = \left[\bar{X} - z_{0.95/2} \cdot \frac{\sigma}{\sqrt{n}} ; \bar{X} + z_{0.95/2} \cdot \frac{\sigma}{\sqrt{n}}\right],$$
    
    Substituindo, temos que? O intervalo de confiança é:
    \[
    IC(\mu, 95\%) = [4,2 - 1,96 \cdot 0,04; 4,2 + 1,96 \cdot 0,04]  = [4.1216; 4.2784].
    \]
    
    \textbf{Interpretação:} Com 95\% de confiança, podemos afirmar que a média do peso dos recém-nascidos no hospital está entre $4,1216$ Kg e $4,2784$ Kg. Ou seja, se obtivéssemos $m$ amostras de tamanho 100 e calculássemos o intervalo de confiança utilizando cada uma das médias amostrais, é esperado que aproximadamente $95\%$ dos intervalos contenham a média populacional. 
    
    \item O raciocínio é semelhante ao do exercício anterior, muda apenas o valor de $z$. Importante apresentar sua observação a respeito da influência do nível de confiança na amplitude do intervalo. 
    \item Raciocínio semelhante ao da questão b, mudando apenas o tamanho amostral. Importante apresentar sua observação a respeito da influência do tamanho da amostra na amplitude do intervalo. 
\end{enumerate}


3)
Para a resolução dos itens abaixo definamos a seguinte variável aleatória:

$$X: \text{ Número de acidentes por dia} $$

Segue que:

$$X \sim Poisson(2)$$
\begin{enumerate}[a)]
    \item  $\mathds{P}(X=0) = \dfrac{e^{-2} 2^0}{0!} = e^{-2} \approx 0,1353 $
    \item \begin{align*}
        \mathds{P}(X \geq 2) &= 1 - \mathds{P}(X < 2)\\
        &=  1- (\mathds{P}(X=0) + \mathds{P}(X=1))\\
        &= 1-\left(e^{-2} + \dfrac{e^{-2} \cdot 2^1}{1!}\right)\\
        &=  1- 3e^{-2} \approx 0,5941
    \end{align*}
    \item Nesse caso definiremos outra variável aleatória $Y$:

    $$Y: \text{ Número de acidentes por semana} $$

    Segue que:
    $$ Y \sim Poisson(2\cdot 7 )$$

    Logo:

    $\mathds{P}(Y \geq 10) = 1 - \mathds{P}(Y < 10) = 1- \sum_{k=0}^{9} \dfrac{e^{-14}14^k}{k!}$

    Obs: Nesse caso é suficiente deixar a expressão dessa forma, não precisa obter um valor numérico. 
\end{enumerate}
\end{document}