\documentclass{article}
\usepackage[shortlabels]{enumitem}
\usepackage{float}

\title{Lista 1 - Estatística para Administração - 2024.2}
\begin{document}
\date{}
\maketitle

1) Classifique cada uma das variáveis abaixo em qualitativa(nominal/ordinal) ou quantitativa (discreta/contínua)

\begin{enumerate}[a)] % a), b), c), ...
\item Ocorrência de hipertensão em adultos maiores de 35 anos (as possíveis respostas são sim ou não)
\item Intenção de voto para presidente (as possíveis respostas são os nomes dos presidentes)
\item Perda de peso de maratonistas na Corrida de São Silvestre em quilos
\item Intensidade da perda de peso de maratonistas na Corrida de São Silvestre (leve, moderada, forte)
\item Grau de satisfação da população com relação ao trabalho de seu presidente (valores de 0 a 5, com 0 indicando totalmente insatisfeito e 5 totalmente satisfeito)
\end{enumerate}

2) Os dados abaixo referem-se ao salário (em salários mínimos) de 20 funcionários administrativos de uma indústria
\begin{table}[H]
\centering
\begin{tabular}{|c|c|c|c|c|c|c|c|c|c|}
\hline
10,1 & 7,3  & 8,5 & 5,0 & 4,2  & 3,1 & 2,2 & 9,0 & 9,4 & 6,1 \\ \hline
3,3  & 10,7 & 1,5 & 8,2 & 10,0 & 4,7 & 3,5 & 6,5 & 8,9 & 6,1 \\ \hline
\end{tabular}
\end{table}

\begin{enumerate}[a)]
\item Construa uma tabela de frequência, agrupando os dados em classes a partir de 1 com amplitude 2. Defina intervalos fechados no limite inferior.
\item Obtenha a média e a mediana dos dados.
\item Obtenha o desvio médio, variância e desvio padrão  dos dados.
\item Obtenha o coeficiente de variação dos dados.
\end{enumerate}

3) Uma pesquisa com usuários de transporte coletivo da cidade do Rio de Janeiro indagou sobre os diferentes meios de transporte utilizados em suas locomoções diárias. Dentre ônibus, metrô e trem, a quantidade de diferentes meios de transporte utilizados foi o seguinte:

$$2, 3, 2, 1, 2, 1, 2, 1, 2, 3, 1, 1, 1, 2, 2, 3, 1, 1, 1, 1, 2, 1, 1, 2, 2, 1, 2,
1 ,2, 3 $$

\begin{enumerate}[a)]
\item Classifique a variável quantidade de diferentes meios de transporte utilizados em qualitativa(nominal/ordinal) ou quantitativa (discreta/contínua).
\item Obtenha a média, a moda e a mediana desse conjunto de valores.
\item Obtenha o desvio médio, variância e desvio padrão desse conjunto de valores.
\item Obtenha o coeficiente de variação desse conjunto de valores. O que podemos dizer sobre a variabilidade desses dados?
\item Organize os dados em uma tabela de frequência.
\end{enumerate}

4) Um questionário foi aplicado aos dez funcionários do setor de contabilidade de uma empresa, fornecendo os dados apresentados na seguinte tabela
\begin{table}[H]
\begin{tabular}{ccccc}
\hline
Funcionário & Curso (completo) & Idade & Salário (R\$) & Anos de Empresa \\ \hline
1           & superior         & 34    & 1440,00       & 5               \\
2           & superior         & 43    & 1450,00       & 8               \\
3           & médio            & 32    & 1440,00       & 6               \\
4           & médio            & 37    & 3300,00       & 8               \\
5           & médio            & 24    & 4300,00       & 3               \\
6           & médio            & 25    & 2700,00       & 2               \\
7           & médio            & 27    & 1960,00       & 5               \\
8           & médio            & 22    & 5600,00       & 2               \\
9           & fundamental      & 21    & 6700,00       & 3               \\
10          & fundamental      & 26    & 4500,00       & 3               \\ \hline
\end{tabular}
\end{table}

\begin{enumerate}[a)]
\item Classifique cada variável em qualitativa(nominal/ordinal) ou quantitativa (discreta/contínua).
\item Obtenha a média, moda e mediana da variável idade.
\item Obtenha o primeiro, segundo (mediana) e terceiro quartis da variável salário.
\item Obtenha o desvio médio e a variância da variável idade.
\item  Construa uma tabela de frequência para a variável Curso (lembre-se de colocar a frequência absoluta e a acumulada)
\item Construa uma tabela de frequência para a variável Idade agrupando em faixas etárias de amplitude 4 incluindo apenas o limite inferior de cada intervalo (lembre-se de colocar a frequência absoluta e a acumulada)
\end{enumerate}

\end{document}