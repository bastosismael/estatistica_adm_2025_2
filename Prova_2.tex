\documentclass{article}
\usepackage[shortlabels]{enumitem}
\usepackage{float}
\usepackage{amsmath}
\usepackage{dsfont}
\usepackage[a4paper, total={6in, 12in}]{geometry}
\title{Prova 2 - Estatística para Administração - 2024.2}
\begin{document}
\date{}
\maketitle

1) (2 pontos) Uma loja virtual recebe, em média, 4 pedidos de um determinado produto por hora. O gerente de operações está avaliando se o estoque é suficiente para atender à demanda sem atrasos. Sabe-se que o número de pedidos por hora segue uma distribuição de Poisson.

\begin{enumerate}[a)]
    \item (0,5 ponto) Qual é a probabilidade de a loja receber exatamente 6 pedidos desse produto em uma hora?
    \item (0,5 ponto) Qual é a probabilidade de receber no máximo 2 pedidos em uma hora?
    \item (1 ponto) A loja está planejando atender 12 horas por dia. Qual é a probabilidade de receber mais de 50 pedidos ao longo de um dia?
\end{enumerate}

\vspace{5px}

2) (3 pontos) Uma empresa de consultoria quer avaliar o impacto médio de um treinamento de liderança sobre o desempenho dos gestores em termos de produtividade (medida em número de projetos concluídos por mês). A variância do número de projetos concluídos mensalmente por gestores é conhecida e igual a \( \sigma^2 = 9 \).  

Para essa análise, foi realizada uma pesquisa utilizando o método de amostragem aleatória simples com 25 gestores que participaram do treinamento. A partir dessa amostra, foi obtido que o número médio de projetos concluídos foi igual a 18.  

\begin{enumerate}[a)]
    \item (0,5 ponto) Explique com suas palavras o método de amostragem aleatória simples. Por que ele é o método mais utilizado na prática?
    \item (0,5 ponto) Calcule um intervalo de confiança de \( 90\% \) para a média do número de projetos concluídos pelos gestores após o treinamento e interprete o resultado encontrado. 
    \item (0,5 ponto) Calcule um intervalo de confiança de \( 95\% \) para a média do número de projetos concluídos pelos gestores após o treinamento e interprete o resultado encontrado.
    \item (0,5 ponto) A empresa conseguiu aumentar o tamanho da amostra para 70 gestores e obteve o mesmo número médio de projetos concluídos. Calcule o  intervalo de confiança de \( 90\% \) para a média com esse novo tamanho de amostra. 
    \item (1 ponto) Explique com suas palavras como o tamanho da amostra e o nível de confiança influenciam na amplitude do intervalo encontrado. 
\end{enumerate}

\vspace{5px}

3) (2 pontos) Suponha que $X$ seja uma variável aleatória que segue a distribuição Normal com média 20 e variância 36. Obtenha as seguintes probabilidades:
\begin{enumerate}[a)]
    \item (0,5 ponto) $\mathds{P}(X \leq 20)$
    \item (0,5 ponto) $\mathds{P}(|X-20| \leq 5)$
    \item (0,5 ponto) $\mathds{P}(X > 15| X>10)$
    \item (0,5 ponto) $x$ tal que $\mathds{P}(X>x)=0,75$
\end{enumerate}



\end{document}