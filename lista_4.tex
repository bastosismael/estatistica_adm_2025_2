\documentclass{article}
\usepackage[shortlabels]{enumitem}
\usepackage{float}
\usepackage{amsmath}
\usepackage{amsfonts}
\usepackage{dsfont}
\title{Lista 4 - Estatística - 2025.1}
\begin{document}
\date{}
\maketitle

1) Um estudante do curso de Engenharia está investigando a qualidade de produção de lápis em uma pequena fábrica local, 
como parte de um projeto sobre aplicações da estatística em contextos reais. Para isso, ele seleciona aleatoriamente 10 lápis da produção diária e verifica,
 com o auxílio de uma régua e uma lupa, se cada lápis está ou não com defeito (por exemplo, ponta quebrada ou madeira rachada).
  O estudante liga para a empresa que fabrica os lapís e pergunta qual a pocentagem média de lapís com defeito produzidos pela empresa, obtendo como
   resposta que 10\% dos lapís produzidos em um dia apresentam algum defeito de fabricação.

\begin{enumerate}[a)]
    \item Defina a variável aleatória \( X_i \) associada ao \( i \)-ésimo lápis testado. Que tipo de variável aleatória é \( X_i \)? Justifique.
    
    \item Defina a variável aleatória \( Y = \sum_{i=1}^{10} X_i \), correspondente ao número total de lápis defeituosos na amostra de 10 lápis. Qual a distribuição de \( Y \)? Justifique.
    
    \item Qual é a probabilidade de exatamente 3 lápis estarem defeituosos?
    
    \item Qual é a probabilidade de ao menos 1 lápis apresentar defeito?

    \item Suponha que o estudante fique extremamente fascinado com sua investigação e decida testar mais 100 lápis.
     Qual o número esperado de lapís com defeito que ele irá encontrar dentre esses 100?
\end{enumerate}

\vspace{5px}

2) Seja $X_1, X_2, \dots, X_{20}$ uma sequência de variáveis aleatórias I.I.D (Independentes e Identicamente Distribuídas), tal que $X_1 \sim Bernoulli(0.5)$
\begin{enumerate}[a)]
    \item Seja $Y=\sum_{i=1}^{20} X_i$. Mostre que $\mathds{E}[Y] = 10$ e $Var(Y) = 5$
    \item Seja $W=X_1 \cdot X_2$. Diga a distribuição de $W$ e escreva a função de probabilidade de $W$.
    \item Seja $V = \prod_{i=1}^n X_i$. Diga a distribuição de $V$ e escreva a função de probabilidade de $V$.
    \item Seja $U = 4 \cdot X_1 - 2$. Qual o suporte de $U$? Qual a esperança (média)  e variância de $U$?
\end{enumerate} 

\vspace{5px}

3) Durante o monitoramento de uma tubulação que transporta ácido sulfúrico concentrado, um engenheiro químico registra, em média, 3 vazamentos microscópicos por mês ao longo de um trecho de 200 metros da tubulação.

Esses vazamentos, embora pequenos, são importantes para o planejamento de manutenção preventiva e monitoramento ambiental. Supõe-se que a ocorrência desses vazamentos segue uma distribuição de Poisson.

\begin{enumerate}[a)]
    \item Defina a variável aleatória \( S \) correspondente ao número de vazamentos microscópicos em um mês. Qual é a distribuição de \( S \)? Qual é seu parâmetro?

    \item Qual é a probabilidade de que ocorram exatamente 2 vazamentos microscópicos em um dado mês?

    \item Qual é a probabilidade de que, em um semana, não ocorra nenhum vazamento?

    \item Qual é a probabilidade de que ocorram mais de 4 vazamentos em um mês?

    \item Suponha que o engenheiro deseje saber a probabilidade de ocorrerem exatamente 6 vazamentos no intervalo de 2 meses. Qual seria essa probabilidade?
\end{enumerate}

\vspace{5px}

4) A duração de um tipo de sensor utilizado em sistemas de alarme segue uma distribuição contínua com densidade, em meses, dada por:
\[
f(x) = 
\begin{cases}
k(x - 1)(5 - x) & \text{se } 1 \leq x \leq 5\\
0 & \text{caso contrário}
\end{cases}
\]

\begin{enumerate}[a)]
    \item Determine o valor da constante \( k \) que torna \( f(x) \) uma função densidade.
    
    \item Qual é a probabilidade de que um sensor dure entre 2 e 4 meses?
    
    \item Qual é o valor esperado  e a variância da vida útil do sensor?
    
    \item Qual é a mediana da distribuição? (Dica: resolva \( P(X \leq m) = 0{,}5 \))
\end{enumerate}

\vspace{5px}

5) Seja $Z$ uma variável aleatória tal que $Z\sim N(0,1)$, calcule:

\begin{enumerate}[a)]
    \item $\mathds{P}(Z>1)$
    \item $\mathds{P}(-1 \leq Z<1)$
    \item  $\mathds{P}(Z>2| Z> 1)$
    \item $\mathds{P}(Z>2| Z< -1)$
    \item $\mathds{P}(Z>2| Z< 3)$
    \item $\mathds{P}(Z<-2)$
    \item $\mathds{P}(-4<Z<4)$
\end{enumerate}

\vspace{5px}

6) Se X é uma variável aleatória normal com parâmetros $\mu=10$ e $\sigma^2=36$, calcule:

\begin{enumerate}[a)]
    \item $\mathds{P}(X>5)$
    \item $\mathds{P}(4<X<16)$
    \item $\mathds{P}(X<8)$
    \item $\mathds{P}(X<20)$
    \item $\mathds{P}(X>16)$
    \item $\mathds{P}(X>5 | X> 2)$
    \item $x$ tal que $\mathds{P}(X>x) = 0.75$
    \item $\mathds{P}(|X-10| \leq 5)$
\end{enumerate}

\vspace{5px}

7)  Suponha que a altura de um homem de 25 anos de idade, em cm, seja uma variável aleatória normal com parâmetros $\mu = 180$ e $\sigma^2=16$. 
Qual a probabilidade de homens de 25 anos de idade ter mais de 1,88 de altura?

\vspace{5px}

8) Suponha $X \sim N(\mu, \sigma)$, encontre:
\begin{enumerate}[a)]
    \item $\mathds{P}(X \geq \mu + 2\sigma)$
    \item $\mathds{P}(|X-\mu| \leq \sigma)$
\end{enumerate}
\vspace{5px}

9) Uma indústria química produz pastilhas catalisadoras cilíndricas usadas em reações de craqueamento catalítico.
 A massa de cada pastilha é uma variável aleatória contínua que segue uma distribuição normal com média
  \( \mu = 50 \) g e desvio padrão \( \sigma = 1{,}5 \) g.

O setor de controle de qualidade estabelece que as pastilhas devem ter massa entre 47 g e 53 g para serem consideradas \textbf{dentro do padrão}.
 Pastilhas fora desse intervalo são \textbf{descartadas}.

\begin{enumerate}[a)]
    \item Seja \( X \) a variável aleatória que representa a massa de uma pastilha. Qual a distribuição de \( X \)?
    
    \item Qual a probabilidade de uma pastilha estar dentro do padrão ?
    
    \item Qual é a proporção de pastilhas que são descartadas?
    
    \item Se forem produzidas 10.000 pastilhas em um lote, qual o número esperado de pastilhas descartadas?
    
    \item O engenheiro decide ajustar o processo para reduzir o desvio padrão para \( \sigma = 1{,}0 \) g, mantendo a média. 
    Qual seria agora a nova proporção esperada de pastilhas dentro do padrão (entre 47 g e 53 g)?
\end{enumerate}

\vspace{5px}


10) Uma empresa de consultoria quer avaliar o impacto médio de um treinamento de liderança sobre o desempenho dos gestores em termos de produtividade
 (medida em número de projetos concluídos por mês). A variância do número de projetos concluídos mensalmente por gestores é conhecida e igual a 
 \( \sigma^2 = 9 \).  

Para essa análise, foi realizada uma pesquisa utilizando o método de amostragem aleatória simples com 25 gestores que participaram do treinamento. A partir dessa amostra, foi obtido que o número médio de projetos concluídos foi igual a 18.  

\begin{enumerate}[a)]
    \item  Calcule um intervalo de confiança de \( 90\% \) para a média do número de projetos concluídos pelos gestores após o treinamento e interprete o resultado encontrado. 
    \item Calcule um intervalo de confiança de \( 95\% \) para a média do número de projetos concluídos pelos gestores após o treinamento e interprete o resultado encontrado.
    \item A empresa conseguiu aumentar o tamanho da amostra para 70 gestores e obteve o mesmo número médio de projetos concluídos. Calcule o  intervalo de confiança de \( 90\% \) para a média com esse novo tamanho de amostra. 
    \item Explique com suas palavras como o tamanho da amostra e o nível de confiança influenciam na amplitude do intervalo encontrado. 
\end{enumerate}

11) Durante o período pré-eleitoral em um município, uma equipe de estatísticos contratada por uma empresa de consultoria em engenharia
 foi encarregada de realizar uma pesquisa de intenção de voto. Foram entrevistados aleatoriamente 625 eleitores.

Dentre os entrevistados, 286 declararam intenção de votar no candidato Luiz Inácio Bolsonaro.

\begin{enumerate}[a)]
    \item Estime a proporção \( \hat{p} \) de eleitores que pretendem votar em Luiz Inácio Bolsonaro com base na amostra.

    \item Construa um intervalo de confiança de 95\% para a verdadeira proporção de eleitores que pretendem votar no candidato.

    \item A legislação eleitoral exige que a margem de erro da pesquisa não ultrapasse 2 pontos percentuais.
     A pesquisa atende a esse critério? Justifique com cálculos.

    \item Qual deveria ser o tamanho mínimo da amostra para garantir uma margem de erro máxima de 2 pontos percentuais com 95\% de confiança,
     independentemente da proporção observada?
\end{enumerate}

\end{document} 