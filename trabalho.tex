\documentclass{article}
\usepackage[shortlabels]{enumitem}
\usepackage{float}
\usepackage{xcolor}
\usepackage[a4paper, total={6in, 10in}]{geometry}
\title{Trabalho - Bioestatística - 2025.1}
\begin{document}
\date{}
\maketitle

\section{Instruções para o Trabalho 1}

Neste trabalho, vocês se posicionarão na figura de um pesquisador que deseja elaborar uma texto (artigo) acerca dos resultados se sua pesquisa.
Ao longo do trabalho, vocês devem usar os conceitos aprendidos nessa primeira parte da disciplina. É importante que vocês se coloquem 
no papel de um pesquisador/investigador e elaborem perguntas acerca do seu tema as quais serão respondidas ao longo de seu texto. 

Para obter o tema e objetivo de sua pesquisa, os seguintes passos devem ocorrer:
\begin{enumerate}
    \item Colocar seu nome e e-mail no tema de pesquisa de interesse. 
    \item Cada grupo deve ser composto de exatamente três pessoas. 
    \item A determinação do grupo se dará de forma aleatória, de acordo com as pessoas que manifestaram interesse no tema. 
    \item Após o período de preenchimento, enviarei um e-mail para cada grupo com uma amostra coletada acerca do tema de pesquisa selecionado. 
    \item Para fins didáticos, a amostra enviada por e-mail será tratada como sendo a população de estudo, uma vez que a etapa de coleta de dados não será realizada por vocês.
\end{enumerate}

O trabalho será avaliado como uma nota entre 0 e 10. As seguintes situações implicarão automaticamente em nota 0:

\begin{itemize}
    \item Entregar o trabalho fora do prazo. 
    \item Entregar o trabalho de forma não-estruturada (desorganizado). 
    \item Não usar a linguagem R.
    \item Não entregar todos os arquivos.
    \item Código não condizente com o texto. 
    \item Não colocar o nome e e-mail no arquivo de seleção de temas durante o período de seleção. 
\end{itemize}

Ao término do trabalho, é necessário que vocês enviem os seguintes itens:

\begin{itemize}
    \item Arquivo em texto da sua pesquisa. 
    \item Código utilizado na pesquisa.  
    \item Imagens adicionais (Caso existam).
\end{itemize}

\textcolor{red}{Obsevações}:
\begin{itemize}
    \item A escrita do código \textbf{não} será avaliada. Entretanto, será verificado se o código está condizente com os resultados do seu texto. 
    \item Caso seu grupo opte por fazer o trabalho usando o Quadro/R Markdown, então o arquivo markdown substitui o código, ou seja, 
    não precisa criar um arquivo com o código separado. 
\end{itemize}

Após a obtenção do seu conjunto de dados, vocês devem realizar a Parte 1 do trabalho. 

\section{Parte 1 - Amostragem e Análise Exploratória de Dados}

Para realização do trabalho, os seguintes elementos devem estar presentes:
Título do trabalho: Deve estar relacionado com a pergunta principal que vocês desejam responder com sua pesquisa. Não precisa ser algo muito rebuscado.  
Capítulo 1 - Introdução: Neste capítulo vocês devem falar sobre aspectos gerais de sua pesquisa. Como por exmplo: Tamanho da população, objetivo da pesquisa, tema da pesquisa e
perguntas a serem respondidas ao longo da pesquisa. Vocês não precisam se limitar a esses exemplos, podendo colocar mais elementos que vocês julgarem importantes. 

Capítulo 2 - Descrição do conjunto de dado: Neste capítulo vocês devem elencar as colunas poresentes no seu conjunto de dados, explicando a informação que cada uma delas traz 
e classificando-as em Quantitativa(Discreta ou Contínua) ou Qualitativa(Nominal ou Ordinal). Caso a variável seja Qualitativa, vocês devem dizer quais os possíveis valores que ela assume e 
o que cada valor significa. 

Capítulo 3 - Análise Exploratória: Nessa seção vocês devem fazer uma análise de todas as colunas presentes no seu conjunto de dados. Para isso, vocês devem utilizar
os gráficos que vimos nas aulas (dispersão, barras, histograma ou boxplot) e as tabelas de frequência. É importante que vocês interpretem cada uma das visualizações geradas, 
valendo-se dessas intepretações para responder perguntas ou gerar insights a respeito do objeto de estudo. 

Capítulo 4 - Conclusão: Nessa seção vocês devem discutir o quwe vocês podem concluir com a pesquisa de vocês e reponder a pergunta definida inicialmente. Nessa 
seção também podem ser levantadas as limitações que impactaram o desenvolvimento do trabalho. 



\textcolor{red}{Importante}: A estrutura acima é apenas o requisito mínimo do trabalho, vocês podem ir além caso julguem necessário (Esse aspecto contribuirá positivamente para sua nota).
É importante que ao longo do trabalho vocês exercitem a criatividade.


\section{Parte 2 - Impacto do tamanho amostral e propriedade da média amostral}

1) Faça os seguintes passos:
\begin{enumerate}[a)]
    \item Selecione uma variável quantitativa do seu banco de dados. 
    \item Calcule a média populacional dessa variável e a armazene em uma variável chamada \textit{media\_populacional}.
    \item Crie dois conjuntos de dados (vetores) vazios no R. Um chamado \textit{media\_amostral} e outro chamado  \textit{variancia\_amostral}.
    \item Obtenha uma amostra de tamanho $5\%$ de $n$ por meio da Amostragem Aleatória Simples do seu conjunto de dados. 
    \item Calcule a média e variância amostral dessa amostra e armazene nos vetores \textit{media\_amostral} e \textit{variancia\_amostral}, respectivamente.
    \item Repita os itens d) e e) para uma amostra de tamanho $10\%, 15\%, 20\%, 25\%, 30\%, 35\%, 40\%, 45\%, 50\%, 55\%, 60\%, 65\%, 70\%, 75\%, 85\%, 90\%, 95\%$ do 
    tamanho amostral.
    \item Construa um gráfico que possua:
    \begin{itemize}
    \item Um gráfico de dispersão com $y=$\textit{media\_amostral} e $x=$\{5, 10, 15, 20, 25, 30, \dots, 95 \} (As porcentagens adotadas no tamanho amostral, sem o símbolo \%)
    \item Na mesma figura, faça outro gráfico de dispersão em que $y=$\textit{variancia\_amostral} e $x=$\{5, 10, 15, 20, 25, 30, \dots, 95 \}. Utilize uma cor diferente da usada anteriormente. 
    \item Uma reta horizontal vermelha em $y=$\textit{media\_populacional}.
    \item Escreva um texto explicando o que você entende ao analisar o gráfico. 
    \end{itemize} 
\end{enumerate}


2) Faça os seguintes passos:

\begin{enumerate}[a)]
    \item Selecione uma variável quantitativa do seu banco de dados (Recomenda-se que seja selecionada a mesma do item anterior). 
    \item Calcule a média populacional dessa variável e a armazene em uma variável chamada \textit{media\_populacional}.
    \item Crie um conjunto de dados (vetores) vazios no R chamado \textit{media\_amostral}.
    \item Obtenha 1000  amostras de tamanho $50\%$ de $n$ por meio da Amostragem Aleatória Simples do seu conjunto de dados e armazene em uma variável chamada \textit{amostras}. 
    \item Calcule a média de cada uma das amostras e armazene-as no vetor \textit{media\_amostras}.
    \item Crie um conjunto de dados (vetor) vazio chamado \textit{media\_das\_medias}.
    \item O vetor \textit{media\_das\_medias} deve ter tamanho 1000, cada posição $i \in \{1,2, 3, 4, 5, \dots, 1000\}$ desse vetor é obtida fazendo a média dos $i$ primeiros elementos do conjunto \textit{media\_das\_medias}. 
    \item Construa um gráfico que possua:
    \begin{itemize}
    \item Um gráfico de dispersão com $y=$\textit{media\_das\_medias} e $x=$\{1, 2, 3, 4, 5 \dots, 1000 \}.
    \item Uma reta horizontal vermelha em $y=$\textit{media\_populacional}.
    \item Escreva um texto explicando que conclusão você tira ao analisar o gráfico. 
    \end{itemize} 
\end{enumerate}

\end{document}